\documentclass[a4paper]{book}
\usepackage[times,inconsolata,hyper]{Rd}
\usepackage{makeidx}
\usepackage[utf8]{inputenc} % @SET ENCODING@
% \usepackage{graphicx} % @USE GRAPHICX@
\makeindex{}
\begin{document}
\chapter*{}
\begin{center}
{\textbf{\huge Package `cffdrs'}}
\par\bigskip{\large \today}
\end{center}
\inputencoding{utf8}
\ifthenelse{\boolean{Rd@use@hyper}}{\hypersetup{pdftitle = {cffdrs: Canadian Forest Fire Danger Rating System}}}{}\ifthenelse{\boolean{Rd@use@hyper}}{\hypersetup{pdfauthor = {Xianli Wang; Alan Cantin; Marc-André Parisien; Mike Wotton; Kerry Anderson; Brett Moore; Tom Schiks; Mike Flannigan}}}{}\begin{description}
\raggedright{}
\item[Type]\AsIs{Package}
\item[Title]\AsIs{Canadian Forest Fire Danger Rating System}
\item[Version]\AsIs{1.8.16}
\item[Date]\AsIs{2020-05-26}
\item[Maintainer]\AsIs{Alan Cantin }\email{Alan.Cantin@Canada.ca}\AsIs{}
\item[Depends]\AsIs{R(>= 3.2.2), rgdal, raster, foreach}
\item[Imports]\AsIs{data.table, geosphere, doParallel}
\item[Description]\AsIs{This project provides a group of new functions to calculate the
outputs of the two main components of the Canadian Forest Fire Danger Rating
System (CFFDRS) Van Wagner and Pickett (1985)
<}\url{https://cfs.nrcan.gc.ca/publications?id=19973}\AsIs{>)
at various time scales: the Fire Weather Index (FWI) System Wan Wagner
(1985) <}\url{https://cfs.nrcan.gc.ca/publications?id=19927}\AsIs{> and the Fire
Behaviour Prediction (FBP) System Forestry Canada Fire Danger Group (1992)
<}\url{http://cfs.nrcan.gc.ca/pubwarehouse/pdfs/10068.pdf}\AsIs{>. Some functions
have two versions, table and raster based.}
\item[License]\AsIs{GPL-2}
\item[URL]\AsIs{}\url{https://r-forge.r-project.org/projects/cffdrs/}\AsIs{}
\item[BugReports]\AsIs{}\url{https://r-forge.r-project.org/tracker/?func=browse&group_id=1970&atid=5372}\AsIs{}
\item[Encoding]\AsIs{UTF-8}
\item[LazyData]\AsIs{true}
\item[RoxygenNote]\AsIs{7.1.2}
\item[Author]\AsIs{Xianli Wang [cre, aut],
Alan Cantin [aut],
Marc-André Parisien [aut],
Mike Wotton [aut],
Kerry Anderson [aut],
Brett Moore [aut],
Tom Schiks [aut],
Mike Flannigan [aut]}
\end{description}
\Rdcontents{\R{} topics documented:}
\inputencoding{utf8}
\HeaderA{cffdrs-package}{Canadian Forest Fire Danger Rating System}{cffdrs.Rdash.package}
\aliasA{cffdrs}{cffdrs-package}{cffdrs}
\keyword{package}{cffdrs-package}
%
\begin{Description}\relax
The cffdrs package allows R users to calculate the outputs of the two main
components of the Canadian Forest Fire Danger Rating System (CFFDRS;
\url{http://cwfis.cfs.nrcan.gc.ca/background/summary/fdr}): the Fire Weather
Index (FWI) System
(\url{http://cwfis.cfs.nrcan.gc.ca/background/summary/fwi}) and the Fire
Behaviour Prediction (FBP) System
(\url{http://cwfis.cfs.nrcan.gc.ca/background/summary/fbp}) along with
additional methods created and used Canadian fire modelling. These systems
are widely used internationally to assess fire danger (FWI System) and
quantify fire behavior (FBP System).
\end{Description}
%
\begin{Details}\relax
The FWI System (Van Wagner 1987) is based on the moisture content and the
effect of wind of three classes of forest fuels on fire behavior. It
consists of six components: three fuel moisture codes (Fire Fuel Moisture
Code, Duff Moisture Code, Drought Code), and three fire behavior indexes
representing rate of spread (Initial Spread Index), fuel consumption
(Buildup Index), and fire intensity (Fire Weather Index). The FWI System
outputs are determined from daily noon weather observations: temperature,
relative humidity, wind speed, and 24-hour rainfall.

The FBP System (Forestry Canada Fire Danger Group 1992; Hirsch 1996)
provides a set of primary and secondary measures of fire behavior. The
primary outputs consist of estimates of fire spread rate, fuel consumption,
fire intensity, and fire description (i.e., surface, intermittent, or crown
fire). The secondary outputs, which are not used nearly as often, give
estimates of fire area, perimeter, perimeter growth rate, and flank and back
fire behavior based on a simple elliptical fire growth model. Unlike the FWI
System, which is weather based, the FBP System also requires information on
vegetation (hereafter, fuel types) and slope (if any) to calculate its
outputs. Sixteen fuel types are included in the FBP System, covering mainly
major vegetation types in Canada.


\Tabular{ll}{ Package: & cffdrs\\{} Type: & Package\\{} Version: &
1.8.16\\{} Date: & 2020-05-26\\{} License: & GPL-2\\{} } This package
includes eleven functions. Seven functions, \code{\LinkA{fwi}{fwi}},
\code{\LinkA{fwiRaster}{fwiRaster}}, \code{\LinkA{hffmc}{hffmc}}, \code{\LinkA{hffmcRaster}{hffmcRaster}},
\code{\LinkA{sdmc}{sdmc}}, \code{\LinkA{gfmc}{gfmc}}, and \code{\LinkA{wDC}{wDC}} are used for
FWI System calculation, whereas two functions, \code{\LinkA{fbp}{fbp}} and
\code{\LinkA{fbpRaster}{fbpRaster}} are used for FBP System calculation. One function,
\code{\LinkA{fireSeason}{fireSeason}} determines fire season start and end dates based on
weather. Two functions \code{\LinkA{pros}{pros}} and \code{\LinkA{lros}{lros}} are rate of
spread and direction calculations across tiangles. These functions are not
fully independent: their inputs overlap greatly and the users will have to
provide FWI System outputs to calculate FBP System outputs. The fwi,
fwiRaster, and sdmc functions calculate the outputs based on daily noon
local standard time (LST) weather observations of temperature, relative
humidity, wind speed, and 24-hour rainfall, as well as the previous day's
moisture content. The hffmc, gfmc, and hffmcRaster functions calculate the
outputs based on hourly weather observations of temperature, relative
humidity, wind speed, and hourly rainfall, as well as the previous hour's
weather conditions. The fbp and fbpRaster functions calculate the outputs of
the FBP System based on given set of information about fire weather
conditions (weather observations and their associated FWI System
components), fuel type, and slope (optional).
\end{Details}
%
\begin{Author}\relax
Xianli Wang, Alan Cantin, Marc-André Parisien, Mike Wotton, Kerry
Anderson, Brett Moore, Tom Schiks, and Mike Flannigan

Maintainer: Alan Cantin \email{Alan.Cantin@Canada.ca}
\end{Author}
%
\begin{References}\relax
1. Van Wagner, C.E. and T.L. Pickett. 1985. Equations and
FORTRAN program for the Canadian Forest Fire Weather Index System. Can. For.
Serv., Ottawa, Ont. For. Tech. Rep. 33. 18 p.

2. Van Wagner, C.E. 1987. Development and structure of the Canadian forest
fire weather index system. Forest Technology Report 35. (Canadian Forestry
Service: Ottawa).

3. Lawson, B.D. and O.B. Armitage. 2008. Weather guide for the Canadian
Forest Fire Danger Rating System. Nat. Resour. Can., Can. For. Serv., North.
For. Cent., Edmonton, AB.

4. Hirsch K.G. 1996. Canadian Forest Fire Behavior Prediction (FBP) System:
user's guide. Nat. Resour. Can., Can. For. Serv., Northwest Reg., North.
For. Cent., Edmonton, Alberta. Spec. Rep. 7. 122p.

5. Forestry Canada Fire Danger Group. 1992. Development and structure of the
Canadian Forest Fire Behavior Prediction System. Forestry Canada, Ottawa,
Ontario Information Report ST-X-3. 63 p.
\url{http://cfs.nrcan.gc.ca/pubwarehouse/pdfs/10068.pdf}

6. Wotton, B.M., Alexander, M.E., Taylor, S.W. 2009. Updates and revisions
to the 1992 Canadian forest fire behavior prediction system. Nat. Resour.
Can., Can. For. Serv., Great Lakes For. Cent., Sault Ste. Marie, Ontario,
Canada. Information Report GLC-X-10, 45p.
\url{http://publications.gc.ca/collections/collection_2010/nrcan/Fo123-2-10-2009-eng.pdf}

7. Tymstra, C., Bryce, R.W., Wotton, B.M., Armitage, O.B. 2009. Development
and structure of Prometheus: the Canadian wildland fire growth simulation
Model. Nat. Resour. Can., Can. For. Serv., North. For. Cent., Edmonton, AB.
Inf. Rep. NOR-X-417.
\end{References}
%
\begin{SeeAlso}\relax
\code{\LinkA{fbp}{fbp}}, \code{\LinkA{fireSeason}{fireSeason}}, \code{\LinkA{fwi}{fwi}},
\code{\LinkA{fwiRaster}{fwiRaster}}, \code{\LinkA{gfmc}{gfmc}}, \code{\LinkA{hffmc}{hffmc}},
\code{\LinkA{hffmcRaster}{hffmcRaster}}, \code{\LinkA{lros}{lros}}, \code{\LinkA{pros}{pros}},
\code{\LinkA{sdmc}{sdmc}}, \code{\LinkA{wDC}{wDC}}
\end{SeeAlso}
%
\begin{Examples}
\begin{ExampleCode}

# Calculating daily FWI with wintering DC 
# 
# This exercise demonstrates how to calculate daily FWI System variables given a 
# chronical two years daily fire weather observations from one weather station.
# In the example, we showed first how to decide fire season start and end 
# dates with fireSeason, we then made overwintering DC adjustment with wDC for
# the second fire season, and eventually calculated the daily FWI System 
# variables over two fire seasons with fwi.  All these steps were packed up 
# into an example user's function, which could be modified by various user 
# groups. Note: the data used in this example is also the test data for wDC.
#  
# library(cffdrs)

#Example of a customised function to calculate fwi and 
#overwinter DC. This could be further modified by 
#users with various needs.
fwi_fs_wDC <- function(input){
  all.fwi <- NULL
  curYr.fwi <- NULL
  #Create date variable
  input$date <- as.Date(as.POSIXlt(paste(input$yr, "-", input$mon, "-", input$day,sep="")))
  
  #use default fire season start and end temperature thresholds
  fs <- fireSeason(input)
  #Fire season dates, ordered chronologically
  fs <- with(fs,fs[order(yr,mon,day),])
  #Create same Date format as weather dataset for comparison
  fs$date <- as.Date(as.POSIXlt(paste(fs$yr,"-",fs$mon,"-",fs$day,sep="")))

  theyears <- unique(fs$yr)
  
  for(curYr.row in 1:length(theyears)){
    curYr <- theyears[curYr.row]
    curYr.d <- fs[fs$yr==curYr,]
    curYr.init <- data.frame(ffmc=80,dmc=10,dc=16) #set an initial startup values
    
    #if there is more than one year of data, accumulate precipitation, then calculate overwinterDC
    #and continue
    if(curYr.row > 1){
      #calculate the overwinter period
      #end of last year's fire season
      curYr.owd <- curYr.fsd[nrow(curYr.fsd),]
      #rbind with beginning of current year's fire season
      curYr.owd <- rbind(curYr.owd, curYr.d[1,])
      
      #accumulate precipitation for the period between end of last and start of current
      curYr.owdata <- sum(input[(input$date>curYr.owd[1,"date"] & 
                          input$date < curYr.owd[2,"date"]),]$prec)
      owDC <- wDC(DCf=tail(curYr.fwi$DC,n=1),rw=curYr.owdata) #calculate overwinter DC value
      curYr.init <- data.frame(ffmc=80,dmc=10,dc=owDC) #Initialize moisture codes
    }    
    
    curYr.fsd <- curYr.d[c(1,nrow(curYr.d)),]#get first and last dates of this year
    #match input data to those dates for fire season data
    curYr.fsdata <- input[input$yr == curYr & input$date >= curYr.fsd[1,"date"] & 
                          input$date <= curYr.fsd[2,"date"],]
    
    #run fwi on fireseason data
    curYr.fwi <- fwi(curYr.fsdata,init=curYr.init)
    #force column names to be uppercase for consistency
    names(curYr.fwi) <- toupper(names(curYr.fwi))
    all.fwi <- rbind(all.fwi,curYr.fwi)
  }
  all.fwi
}

##Usage of the custom function
# Load the test dataset, which is also the test data for wDC:
data("test_wDC")
#select 1 weather station
localWX_1 <- test_wDC[test_wDC$id==1,]
#run function with the data and fire season values
fwi_withFSwDC <- fwi_fs_wDC(localWX_1)
#Check the resulting fwi indices, calculated with a fire season start and end date, and using 
#overwintered DC
fwi_withFSwDC

\end{ExampleCode}
\end{Examples}
\inputencoding{utf8}
\HeaderA{fbp}{Fire Behavior Prediction System function}{fbp}
\keyword{methods}{fbp}
%
\begin{Description}\relax
\code{fbp} calculates the outputs from the Canadian Forest Fire Behavior
Prediction (FBP) System (Forestry Canada Fire Danger Group 1992) based on
given fire weather and fuel moisture conditions (from the Canadian Forest
Fire Weather Index (FWI) System (Van Wagner 1987)), fuel type, date, and
slope. Fire weather, for the purpose of FBP System calculation, comprises
observations of 10 m wind speed and direction at the time of the fire, and
two associated outputs from the Fire Weather Index System, the Fine Fuel
Moisture Content (FFMC) and Buildup Index (BUI). FWI System components can
be calculated with the sister function \code{\LinkA{fwi}{fwi}}.
\end{Description}
%
\begin{Usage}
\begin{verbatim}
fbp(input = NULL, output = "Primary", m = NULL, cores = 1)
\end{verbatim}
\end{Usage}
%
\begin{Arguments}
\begin{ldescription}
\item[\code{input}] The input data, a data.frame containing fuel types, fire
weather component, and slope (see below). Each vector of inputs defines a
single FBP System prediction for a single fuel type and set of weather
conditions. The data.frame can be used to evaluate the FBP System for a
single fuel type and instant in time, or multiple records for a single point
(e.g., one weather station, either hourly or daily for instance) or multiple
points (multiple weather stations or a gridded surface). All input variables
have to be named as listed below, but they are case insensitive, and do not
have to be in any particular order. Fuel type is of type character; other
arguments are numeric. Missing values in numeric variables could either be
assigned as NA or leave as blank.\\{}\\{}



\Tabular{lll}{ 
\bold{Required Inputs:}&&\\{} 
\bold{Input} & \bold{Description/Full name} & \bold{Defaults}\\{} 

\var{FuelType} 
& FBP System Fuel Type including "C-1",\\{}
&"C-2", "C-3", "C-4","C-5", "C-6", "C-7",\\{}
& "D-1", "M-1", "M-2", "M-3", "M-4", "NF",\\{}
& "D-1", "S-2", "S-3", "O-1a", "O-1b", and\\{}
&  "WA", where "WA" and "NF" stand for \\{}
& "water" and "non-fuel", respectively.\\{}\\{}

\var{LAT} & Latitude [decimal degrees] & 55\\{} 
\var{LONG} & Longitude [decimal degrees] & -120\\{} 
\var{FFMC} & Fine fuel moisture code [FWI System component] & 90\\{} 
\var{BUI} & Buildup index [FWI System component] & 60\\{} 
\var{WS} & Wind speed [km/h] & 10\\{}
\var{GS} & Ground Slope [percent] & 0\\{} 
\var{Dj} & Julian day & 180\\{} 
\var{Aspect} & Aspect of the slope [decimal degrees] & 0\\{}\\{} 

\bold{Optional Inputs (1):}
& Variables associated with certain fuel \\{}
& types. These could be skipped if relevant \\{}
& fuel types do not appear in the input data.\\{}\\{}

\bold{Input} & \bold{Full names of inputs} & \bold{Defaults}\\{} 

\var{PC} & Percent Conifer for M1/M2 [percent] & 50\\{} 
\var{PDF} & Percent Dead Fir for M3/M4 [percent] & 35\\{}
\var{cc} & Percent Cured for O1a/O1b [percent] & 80\\{} 
\var{GFL} & Grass Fuel Load [kg/m\textasciicircum{}2] & 0.35\\{}\\{} 

\bold{Optional Inputs (2):} 
& Variables that could be ignored without \\{}
& causing major impacts to the primary outputs\\{}\\{}

\bold{Input} & \bold{Full names of inputs} & \bold{Defaults}\\{} 
\var{CBH}   & Crown to Base Height [m] & 3\\{} 
\var{WD}    & Wind direction [decimal degrees] & 0\\{} 
\var{Accel} & Acceleration: 1 = point, 0 = line & 0\\{} 
\var{ELV*}  & Elevation [meters above sea level] & NA\\{} 
\var{BUIEff}& Buildup Index effect: 1=yes, 0=no & 1\\{} 
\var{D0}    & Julian day of minimum Foliar Moisture Content & 0\\{} 
\var{hr}    & Hours since ignition & 1\\{} 
\var{ISI}   & Initial spread index & 0\\{} 
\var{CFL}   & Crown Fuel Load [kg/m\textasciicircum{}2]& 1.0\\{} 
\var{FMC}   & Foliar Moisture Content if known [percent] & 0\\{} 
\var{SH}    & C-6 Fuel Type Stand Height [m] & 0\\{} 
\var{SD}    & C-6 Fuel Type Stand Density [stems/ha] & 0\\{} 
\var{theta} & Elliptical direction of calculation [degrees] & 0\\{}\\{} }

\item[\code{output}] FBP output offers 3 options (see details in \bold{Values}
section):


\Tabular{lc}{ \bold{Outputs} & \bold{Number of outputs}\\{} 
\var{Primary(\bold{default})} & 8\\{} 
\var{Secondary} & 34\\{} 
\var{All} & 42\\{}\\{}}

\item[\code{m}] Optimal number of pixels at each iteration of computation when
\code{nrow(input) >= 1000}. Default \code{m = NULL}, where the function will
assign \code{m = 1000} when \code{nrow(input)} is between 1000 and 500,000,
and \code{m = 3000} otherwise. By including this option, the function is
able to process large dataset more efficiently. The optimal value may vary
with different computers.

\item[\code{cores}] Number of CPU cores (integer) used in the computation, default
is 1.  By signing \code{cores > 1}, the function will apply parallel
computation technique provided by the \code{foreach} package, which
significantly reduces the computation time for large input data (over a
million records). For small dataset, \code{cores=1} is actually faster.
\end{ldescription}
\end{Arguments}
%
\begin{Details}\relax
The Canadian Forest Fire Behavior Prediction (FBP) System (Forestry Canada
Fire Danger Group 1992) is a subsystem of the Canadian Forest Fire Danger
Rating System, which also includes the Canadian Forest Fire Weather Index
(FWI) System. The FBP System provides quantitative estimates of head fire
spread rate, fuel consumption, fire intensity, and a basic fire description
(e.g., surface, crown) for 16 different important forest and rangeland types
across Canada. Using a simple conceptual model of the growth of a point
ignition as an ellipse through uniform fuels and under uniform weather
conditions, the system gives, as a set of secondary outputs, estimates of
flank and back fire behavior and consequently fire area perimeter length and
growth rate.

The FBP System evolved since the mid-1970s from a series of regionally
developed burning indexes to an interim edition of the nationally develop
FBP system issued in 1984. Fire behavior models for spread rate and fuel
consumption were derived from a database of over 400 experimental, wild and
prescribed fire observations. The FBP System, while providing quantitative
predictions of expected fire behavior is intended to supplement the
experience and judgment of operational fire managers (Hirsch 1996).

The FBP System was updated with some minor corrections and revisions in 2009
(Wotton et al. 2009) with several additional equations that were initially
not included in the system. This fbp function included these updates and
corrections to the original equations and provides a complete suite of fire
behavior prediction variables. Default values of optional input variables
provide a reasonable mid-range setting. Latitude, longitude, elevation, and
the date are used to calculate foliar moisture content, using a set of
models defined in the FBP System; note that this latitude/longitude-based
function is only valid for Canada. If the Foliar Moisture Content (FMC) is
specified directly as an input, the fbp function will use this value
directly rather than calculate it. This is also true of other input
variables.

Note that Wind Direction (WD) is the compass direction from which wind is
coming. Wind azimuth (not an input) is the direction the wind is blowing to
and is 180 degrees from wind direction; in the absence of slope, the wind
azimuth is coincident with the direction the head fire will travel (the
spread direction azimuth, RAZ). Slope aspect is the main compass direction
the slope is facing. Slope azimuth (not an input) is the direction a head
fire will spread up slope (in the absence of wind effects) and is 180
degrees from slope aspect (Aspect).  Wind direction and slope aspect are the
commonly used directional identifiers when specifying wind and slope
orientation respectively.  The input theta specifies an angle (given as a
compass bearing) at which a user is interested in fire behavior predictions;
it is typically some angle off of the final spread rate direction since if
for instance theta=RAZ (the final spread azimuth of the fire) then the rate
of spread at angle theta (TROS) will be equivalent to ROS.
\end{Details}
%
\begin{Value}
\code{fbp} returns a dataframe with primary, secondary, or all
output variables, a combination of the primary and secondary outputs.

\bold{Primary} FBP output includes the following 8 variables: 

\begin{ldescription}
\item[\code{CFB}] Crown Fraction Burned by the head fire
\item[\code{CFC}] Crown Fuel Consumption [kg/m\textasciicircum{}2]
\item[\code{FD}] Fire description (1=Surface, 2=Intermittent, 3=Crown)
\item[\code{HFI}] Head Fire Intensity [kW/m]
\item[\code{RAZ}] Spread direction azimuth [degrees]
\item[\code{ROS}] Equilibrium Head Fire Rate of Spread [m/min]
\item[\code{SFC}] Surface Fuel Consumption [kg/m\textasciicircum{}2]
\item[\code{TFC}] Total Fuel Consumption [kg/m\textasciicircum{}2]

\end{ldescription}
\bold{Secondary} FBP System outputs include the following 34 raster layers. In order
to calculate the reliable secondary outputs, depending on the outputs, 
optional inputs may have to be provided.  

\begin{ldescription}
\item[\code{BE}] BUI effect on spread rate
\item[\code{SF}] Slope Factor (multiplier for ROS increase upslope)
\item[\code{ISI}] Initial Spread Index
\item[\code{FFMC}] Fine fuel moisture code [FWI System component]
\item[\code{FMC}] Foliar Moisture Content [\%]
\item[\code{Do}] Julian Date of minimum FMC
\item[\code{RSO}] Critical spread rate for crowning [m/min]
\item[\code{CSI}] Critical Surface Intensity for crowning [kW/m]
\item[\code{FROS}] Equilibrium Flank Fire Rate of Spread [m/min]
\item[\code{BROS}] Equilibrium Back Fire Rate of Spread [m/min]
\item[\code{HROSt}] Head Fire Rate of Spread at time hr [m/min]
\item[\code{FROSt}] Flank Fire Rate of Spread at time hr [m/min]
\item[\code{BROSt}] Back Fire Rate of Spread at time hr [m/min]
\item[\code{FCFB}] Flank Fire Crown Fraction Burned
\item[\code{BCFB}] Back Fire Crown Fraction Burned
\item[\code{FFI}] Equilibrium Spread Flank Fire Intensity [kW/m]
\item[\code{BFI}] Equilibrium Spread Back Fire Intensity [kW/m]
\item[\code{FTFC}] Flank Fire Total Fuel Consumption [kg/m\textasciicircum{}2] 
\item[\code{BTFC}] Back Fire Total Fuel Consumption [kg/m\textasciicircum{}2] 
\item[\code{DH}] Head Fire Spread Distance after time hr [m] 
\item[\code{DB}] Back Fire Spread Distance after time hr [m] 
\item[\code{DF}] Flank Fire Spread Distance after time hr [m] 
\item[\code{TI}] Time to Crown Fire Initiation [hrs since ignition] 
\item[\code{FTI}] Time to Flank Fire Crown initiation [hrs since ignition]
\item[\code{BTI}] Time to Back Fire Crown initiation [hrs since ignition]
\item[\code{LB}] Length to Breadth ratio
\item[\code{LBt}] Length to Breadth ratio after elapsed time hr 
\item[\code{WSV}] Net vectored wind speed [km/hr]
\item[\code{TROS*}] Equilibrium Rate of Spread at bearing theta [m/min] 
\item[\code{TROSt*}] Rate of Spread at bearing theta at time t [m/min] 
\item[\code{TCFB*}] Crown Fraction Burned at bearing theta 
\item[\code{TFI*}] Fire Intensity at bearing theta [kW/m] 
\item[\code{TTFC*}] Total Fuel Consumption at bearing theta [kg/m\textasciicircum{}2] 
\item[\code{TTI*}] Time to Crown Fire initiation at bearing theta [hrs since ignition] 

\end{ldescription}
*These outputs represent fire behaviour at a point on the perimeter of an
elliptical fire defined by a user input angle theta. theta represents the
bearing of a line running between the fire ignition point and a point on the
perimeter of the fire. It is important to note that in this formulation the
theta is a bearing and does not represent the angle from the semi-major axis
(spread direction) of the ellipse. This formulation is similar but not
identical to methods presented in Wotton et al (2009) and Tymstra et al
(2009).
\end{Value}
%
\begin{Author}\relax
Xianli Wang, Alan Cantin, Marc-André Parisien, Mike Wotton, Kerry
Anderson, and Mike Flannigan
\end{Author}
%
\begin{References}\relax
1.  Hirsch K.G. 1996. Canadian Forest Fire Behavior Prediction
(FBP) System: user's guide. Nat. Resour. Can., Can. For. Serv., Northwest
Reg., North. For. Cent., Edmonton, Alberta. Spec. Rep. 7. 122p.

2.  Forestry Canada Fire Danger Group. 1992. Development and structure of
the Canadian Forest Fire Behavior Prediction System. Forestry Canada,
Ottawa, Ontario Information Report ST-X-3. 63 p.
\url{http://cfs.nrcan.gc.ca/pubwarehouse/pdfs/10068.pdf}

3.  Wotton, B.M., Alexander, M.E., Taylor, S.W. 2009. Updates and revisions
to the 1992 Canadian forest fire behavior prediction system. Nat. Resour.
Can., Can. For. Serv., Great Lakes For. Cent., Sault Ste. Marie, Ontario,
Canada. Information Report GLC-X-10, 45p.
\url{http://publications.gc.ca/collections/collection_2010/nrcan/Fo123-2-10-2009-eng.pdf}

4.  Tymstra, C., Bryce, R.W., Wotton, B.M., Armitage, O.B. 2009. Development
and structure of Prometheus: the Canadian wildland fire growth simulation
Model. Nat. Resour. Can., Can. For. Serv., North. For. Cent., Edmonton, AB.
Inf. Rep. NOR-X-417.\url{https://d1ied5g1xfgpx8.cloudfront.net/pdfs/31775.pdf}
\end{References}
%
\begin{SeeAlso}\relax
\code{\LinkA{fwi}{fwi}, \LinkA{fbpRaster}{fbpRaster}}
\end{SeeAlso}
%
\begin{Examples}
\begin{ExampleCode}

library(cffdrs)
# The dataset is the standard test data for FPB system
# provided by Wotton et al (2009)
data("test_fbp")
head(test_fbp)
#  id FuelType LAT LONG ELV FFMC BUI   WS WD GS  Dj  D0         hr PC PDF GFL cc theta Accel Aspect BUIEff CBH CFL ISI
#1  1      C-1  55  110  NA   90 130 20.0  0 15 182  NA 0.33333333 NA  NA  NA  NA     0     1    270      1  NA  NA   0
#2  2       C2  50   90  NA   97 119 20.4  0 75 121  NA 0.33333333 NA  NA  NA  NA     0     1    315      1  NA  NA   0
#3  3      C-3  55  110  NA   95  30 50.0  0  0 182  NA 0.08333333 NA  NA  NA  NA     0     1    180      1  NA  NA   0
#4  4      C-4  55  105 200   85  82  0.0 NA 75 182  NA 0.50000000 NA  NA  NA  NA     0     1    315      1  NA  NA   0
#5  5       c5  55  105  NA   88  56  3.4  0 23 152 145 0.50000000 NA  NA  NA  NA     0     1    180      1  NA  NA   0

#Primary output (default)
fbp(test_fbp)
#or
fbp(test_fbp,output="Primary") 
#or 
fbp(test_fbp,"P")
#Secondary output          
fbp(test_fbp,"Secondary")
#or
fbp(test_fbp,"S")
#All output          
fbp(test_fbp,"All")
#or
fbp(test_fbp,"A")
#For a single record:
fbp(test_fbp[7,])  	
#For a section of the records:
fbp(test_fbp[8:13,])	
#fbp function produces the default values if no data is fed to
#the function:
fbp()

\end{ExampleCode}
\end{Examples}
\inputencoding{utf8}
\HeaderA{fbpRaster}{Raster-based Fire Behavior Prediction System Calculations}{fbpRaster}
\keyword{methods}{fbpRaster}
%
\begin{Description}\relax
\code{fbpRaster} calculates the outputs from the Canadian Forest Fire
Behavior Prediction (FBP) System (Forestry Canada Fire Danger Group 1992)
based on raster format fire weather and fuel moisture conditions (from the
Canadian Forest Fire Weather Index (FWI) System (Van Wagner 1987)), fuel
type, date, and slope. Fire weather, for the purpose of FBP System
calculation, comprises observations of 10 m wind speed and direction at the
time of the fire, and two associated outputs from the Fire Weather Index
System, the Fine Fuel Moisture Content (FFMC) and Buildup Index (BUI).
Raster-based FWI System components can be calculated with the sister
function \code{\LinkA{fwiRaster}{fwiRaster}}.
\end{Description}
%
\begin{Usage}
\begin{verbatim}
fbpRaster(input, output = "Primary", select = NULL, m = NULL, cores = 1)
\end{verbatim}
\end{Usage}
%
\begin{Arguments}
\begin{ldescription}
\item[\code{input}] The input data, a RasterStack containing fuel types, fire
weather component, and slope layers (see below). Each vector of inputs
defines a single FBP System prediction for a single fuel type and set of
weather conditions. The RasterStack can be used to evaluate the FBP System
for a single fuel type and instant in time, or multiple records for a single
point (e.g., one weather station, either hourly or daily for instance) or
multiple points (multiple weather stations or a gridded surface). All input
variables have to be named as listed below, but they are case insensitive,
and do not have to be in any particular order. Fuel type is of type
character; other arguments are numeric. Missing values in numeric variables
could either be assigned as NA or leave as blank.


\Tabular{lll}{ 
\bold{Required Inputs:}&&\\{} 
\bold{Input} & \bold{Description/Full name} & \bold{Defaults}\\{} 

\var{FuelType} 
& FBP System Fuel Type including "C-1",\\{}
&"C-2", "C-3", "C-4","C-5", "C-6", "C-7",\\{}
& "D-1", "M-1", "M-2", "M-3", "M-4", "NF",\\{}
& "D-1", "S-2", "S-3", "O-1a", "O-1b", and\\{}
&  "WA", where "WA" and "NF" stand for \\{}
& "water" and "non-fuel", respectively.\\{}\\{}

\var{LAT} & Latitude [decimal degrees] & 55\\{} 
\var{LONG} & Longitude [decimal degrees] & -120\\{} 
\var{FFMC} & Fine fuel moisture code [FWI System component] & 90\\{} 
\var{BUI} & Buildup index [FWI System component] & 60\\{} 
\var{WS} & Wind speed [km/h] & 10\\{}
\var{GS} & Ground Slope [percent] & 0\\{} 
\var{Dj} & Julian day & 180\\{} 
\var{Aspect} & Aspect of the slope [decimal degrees] & 0\\{}\\{} 

\bold{Optional Inputs (1):}
& Variables associated with certain fuel \\{}
& types. These could be skipped if relevant \\{}
& fuel types do not appear in the input data.\\{}\\{}

\bold{Input} & \bold{Full names of inputs} & \bold{Defaults}\\{} 

\var{PC} & Percent Conifer for M1/M2 [percent] & 50\\{} 
\var{PDF} & Percent Dead Fir for M3/M4 [percent] & 35\\{}
\var{cc} & Percent Cured for O1a/O1b [percent] & 80\\{} 
\var{GFL} & Grass Fuel Load [kg/m\textasciicircum{}2] & 0.35\\{}\\{} 

\bold{Optional Inputs (2):} 
& Variables that could be ignored without \\{}
& causing major impacts to the primary outputs\\{}\\{}

\bold{Input} & \bold{Full names of inputs} & \bold{Defaults}\\{} 
\var{CBH}   & Crown to Base Height [m] & 3\\{} 
\var{WD}    & Wind direction [decimal degrees] & 0\\{} 
\var{Accel} & Acceleration: 1 = point, 0 = line & 0\\{} 
\var{ELV*}  & Elevation [meters above sea level] & NA\\{} 
\var{BUIEff}& Buildup Index effect: 1=yes, 0=no & 1\\{} 
\var{D0}    & Julian day of minimum Foliar Moisture Content & 0\\{} 
\var{hr}    & Hours since ignition & 1\\{} 
\var{ISI}   & Initial spread index & 0\\{} 
\var{CFL}   & Crown Fuel Load [kg/m\textasciicircum{}2]& 1.0\\{} 
\var{FMC}   & Foliar Moisture Content if known [percent] & 0\\{} 
\var{SH}    & C-6 Fuel Type Stand Height [m] & 0\\{} 
\var{SD}    & C-6 Fuel Type Stand Density [stems/ha] & 0\\{} 
\var{theta} & Elliptical direction of calculation [degrees] & 0\\{}\\{} }

\item[\code{output}] FBP output offers 3 options (see details in \bold{Values}
section):


\Tabular{lc}{ \bold{Outputs} & \bold{Number of outputs}\\{} \var{Primary
(\bold{default})} & 8\\{} \var{Secondary} & 34\\{} \var{All} & 42\\{}
}

\item[\code{select}] Selected outputs

\item[\code{m}] Optimal number of pixels at each iteration of computation when
\code{ncell(input) >= 1000}. Default m = NULL, where the function will
assign m = 1000 when \code{ncell(input)} is between 1000 and 500,000, and
m=3000 otherwise. By including this option, the function is able to process
large dataset more efficiently. The optimal value may vary with different
computers.

\item[\code{cores}] Number of CPU cores (integer) used in the computation, default
is 1.  By signing \code{cores > 1}, the function will apply parallel
computation technique provided by the \code{foreach} package, which
significantly reduces the computation time for large input data (over a
million grid points). For small dataset, \code{cores=1} is actually faster.
\end{ldescription}
\end{Arguments}
%
\begin{Details}\relax
The Canadian Forest Fire Behavior Prediction (FBP) System (Forestry Canada
Fire Danger Group 1992) is a subsystem of the Canadian Forest Fire Danger
Rating System, which also includes the Canadian Forest Fire Weather Index
(FWI) System. The FBP System provides quantitative estimates of head fire
spread rate, fuel consumption, fire intensity, and a basic fire description
(e.g., surface, crown) for 16 different important forest and rangeland types
across Canada. Using a simple conceptual model of the growth of a point
ignition as an ellipse through uniform fuels and under uniform weather
conditions, the system gives, as a set of secondary outputs, estimates of
flank and back fire behavior and consequently fire area perimeter length and
growth rate.

The FBP System evolved since the mid-1970s from a series of regionally
developed burning indexes to an interim edition of the nationally develop
FBP system issued in 1984. Fire behavior models for spread rate and fuel
consumption were derived from a database of over 400 experimental, wild and
prescribed fire observations. The FBP System, while providing quantitative
predictions of expected fire behavior is intended to supplement the
experience and judgment of operational fire managers (Hirsch 1996).

The FBP System was updated with some minor corrections and revisions in 2009
(Wotton et al. 2009) with several additional equations that were initially
not included in the system. This fbp function included these updates and
corrections to the original equations and provides a complete suite of fire
behavior prediction variables. Default values of optional input variables
provide a reasonable mid-range setting. Latitude, longitude, elevation, and
the date are used to calculate foliar moisture content, using a set of
models defined in the FBP System; note that this latitude/longitude-based
function is only valid for Canada. If the Foliar Moisture Content (FMC) is
specified directly as an input, the fbp function will use this value
directly rather than calculate it. This is also true of other input
variables.

Note that Wind Direction (WD) is the compass direction from which wind is
coming. Wind azimuth (not an input) is the direction the wind is blowing to
and is 180 degrees from wind direction; in the absence of slope, the wind
azimuth is coincident with the direction the head fire will travel (the
spread direction azimuth, RAZ). Slope aspect is the main compass direction
the slope is facing. Slope azimuth (not an input) is the direction a head
fire will spread up slope (in the absence of wind effects) and is 180
degrees from slope aspect (Aspect).  Wind direction and slope aspect are the
commonly used directional identifiers when specifying wind and slope
orientation respectively.  The input theta specifies an angle (given as a
compass bearing) at which a user is interested in fire behavior predictions;
it is typically some angle off of the final spread rate direction since if
for instance theta=RAZ (the final spread azimuth of the fire) then the rate
of spread at angle theta (TROS) will be equivalent to ROS.

Because raster format data cannot hold characters, we have to code these fuel
types in numeric codes. In sequence, the codes are c(1:19). FuelType could 
also be converted as factor and assigned to the raster layer, the function 
will still work.


\Tabular{ll}{
\bold{Fuel Type} & \bold{code} \\{} 
\AsIs{C-1}       & 1           \\{} 
\AsIs{C-2}       & 2           \\{} 
\AsIs{C-3}       & 3           \\{}
\AsIs{C-4}       & 4           \\{} 
\AsIs{C-5}       & 5           \\{} 
\AsIs{C-6}       & 6           \\{} 
\AsIs{C-7}       & 7           \\{} 
\AsIs{D-1}       & 8           \\{} 
\AsIs{M-1}       & 9           \\{} 
\AsIs{M-2}       & 10          \\{} 
\AsIs{M-3}       & 11          \\{} 
\AsIs{M-4}       & 12          \\{} 
\AsIs{NF}        & 13          \\{} 
\AsIs{O-1a}      & 14          \\{} 
\AsIs{O-1b}      & 15          \\{} 
\AsIs{S-1}       & 16          \\{} 
\AsIs{S-2}       & 17          \\{} 
\AsIs{S-3}       & 18          \\{} 
\AsIs{WA}        & 19          \\{}\\{}}
\end{Details}
%
\begin{Value}
Either Primary, Secondary, or all FBP outputs in a raster stack

\code{fbpRaster} returns a RasterStack with primary, secondary, or 
all output variables, a combination of the primary and secondary outputs. 
Primary FBP output includes the following 8 raster layers: 

\begin{ldescription}
\item[\code{CFB}] Crown Fraction Burned by the head fire
\item[\code{CFC}] Crown Fuel Consumption [kg/m\textasciicircum{}2]
\item[\code{FD}] Fire description (1=Surface, 2=Intermittent, 3=Crown)
\item[\code{HFI}] Head Fire Intensity [kW/m]
\item[\code{RAZ}] Spread direction azimuth [degrees]
\item[\code{ROS}] Equilibrium Head Fire Rate of Spread [m/min]
\item[\code{SFC}] Surface Fuel Consumption [kg/m\textasciicircum{}2]
\item[\code{TFC}] Total Fuel Consumption [kg/m\textasciicircum{}2]

\end{ldescription}
Secondary FBP System outputs include the following 34 raster layers. In order 
to calculate the reliable secondary outputs, depending on the outputs, 
optional inputs may have to be provided.  

\begin{ldescription}
\item[\code{BE}] BUI effect on spread rate
\item[\code{SF}] Slope Factor (multiplier for ROS increase upslope)
\item[\code{ISI}] Initial Spread Index
\item[\code{FFMC}] Fine fuel moisture code [FWI System component]
\item[\code{FMC}] Foliar Moisture Content [\%]
\item[\code{Do}] Julian Date of minimum FMC
\item[\code{RSO}] Critical spread rate for crowning [m/min]
\item[\code{CSI}] Critical Surface Intensity for crowning [kW/m]
\item[\code{FROS}] Equilibrium Flank Fire Rate of Spread [m/min]
\item[\code{BROS}] Equilibrium Back Fire Rate of Spread [m/min]
\item[\code{HROSt}] Head Fire Rate of Spread at time hr [m/min]
\item[\code{FROSt}] Flank Fire Rate of Spread at time hr [m/min]
\item[\code{BROSt}] Back Fire Rate of Spread at time hr [m/min]
\item[\code{FCFB}] Flank Fire Crown Fraction Burned
\item[\code{BCFB}] Back Fire Crown Fraction Burned
\item[\code{FFI}] Equilibrium Spread Flank Fire Intensity [kW/m]
\item[\code{BFI}] Equilibrium Spread Back Fire Intensity [kW/m]
\item[\code{FTFC}] Flank Fire Total Fuel Consumption [kg/m\textasciicircum{}2] 
\item[\code{BTFC}] Back Fire Total Fuel Consumption [kg/m\textasciicircum{}2] 
\item[\code{DH}] Head Fire Spread Distance after time hr [m] 
\item[\code{DB}] Back Fire Spread Distance after time hr [m] 
\item[\code{DF}] Flank Fire Spread Distance after time hr [m] 
\item[\code{TI}] Time to Crown Fire Initiation [hrs since ignition] 
\item[\code{FTI}] Time to Flank Fire Crown initiation [hrs since ignition]
\item[\code{BTI}] Time to Back Fire Crown initiation [hrs since ignition]
\item[\code{LB}] Length to Breadth ratio
\item[\code{LBt}] Length to Breadth ratio after elapsed time hr 
\item[\code{WSV}] Net vectored wind speed [km/hr]
\item[\code{TROS*}] Equilibrium Rate of Spread at bearing theta [m/min] 
\item[\code{TROSt*}] Rate of Spread at bearing theta at time t [m/min] 
\item[\code{TCFB*}] Crown Fraction Burned at bearing theta 
\item[\code{TFI*}] Fire Intensity at bearing theta [kW/m] 
\item[\code{TTFC*}] Total Fuel Consumption at bearing theta [kg/m\textasciicircum{}2] 
\item[\code{TTI*}] Time to Crown Fire initiation at bearing theta [hrs since ignition] 

\end{ldescription}
*These outputs represent fire behaviour at a point on the perimeter of an
elliptical fire defined by a user input angle theta. theta represents the
bearing of a line running between the fire ignition point and a point on the
perimeter of the fire. It is important to note that in this formulation the
theta is a bearing and does not represent the angle from the semi-major axis
(spread direction) of the ellipse. This formulation is similar but not
identical to methods presented in Wotton et al (2009) and Tymstra et al
(2009).
\end{Value}
%
\begin{Author}\relax
Xianli Wang, Alan Cantin, Marc-André Parisien, Mike Wotton, Kerry
Anderson, and Mike Flannigan
\end{Author}
%
\begin{References}\relax
1.  Hirsch K.G. 1996. Canadian Forest Fire Behavior Prediction
(FBP) System: user's guide. Nat. Resour. Can., Can. For. Serv., Northwest
Reg., North. For. Cent., Edmonton, Alberta. Spec. Rep. 7. 122p.

2.  Forestry Canada Fire Danger Group. 1992. Development and structure of
the Canadian Forest Fire Behavior Prediction System. Forestry Canada,
Ottawa, Ontario Information Report ST-X-3. 63 p.
\url{http://cfs.nrcan.gc.ca/pubwarehouse/pdfs/10068.pdf}

3.  Wotton, B.M., Alexander, M.E., Taylor, S.W. 2009. Updates and revisions
to the 1992 Canadian forest fire behavior prediction system. Nat. Resour.
Can., Can. For. Serv., Great Lakes For. Cent., Sault Ste. Marie, Ontario,
Canada. Information Report GLC-X-10, 45p.
\url{http://publications.gc.ca/collections/collection_2010/nrcan/Fo123-2-10-2009-eng.pdf}

4.  Tymstra, C., Bryce, R.W., Wotton, B.M., Armitage, O.B. 2009. Development
and structure of Prometheus: the Canadian wildland fire growth simulation
Model. Nat. Resour. Can., Can. For. Serv., North. For. Cent., Edmonton, AB.
Inf. Rep. NOR-X-417.\url{https://d1ied5g1xfgpx8.cloudfront.net/pdfs/31775.pdf}
\end{References}
%
\begin{SeeAlso}\relax
\code{\LinkA{fbp}{fbp}, \LinkA{fwiRaster}{fwiRaster}, \LinkA{hffmcRaster}{hffmcRaster}}
\end{SeeAlso}
%
\begin{Examples}
\begin{ExampleCode}

# The dataset is the standard test data for FBP system
# provided by Wotton et al (2009), and randomly assigned
# to a stack of raster layers
test_fbpRaster <- stack(system.file("extdata", "test_fbpRaster.tif", package="cffdrs"))
input<-test_fbpRaster
# Stack doesn't hold the raster layer names, we have to assign
# them:
names(input)<-c("FuelType","LAT","LONG","ELV","FFMC","BUI", "WS","WD","GS","Dj","D0","hr","PC",
"PDF","GFL","cc","theta","Accel","Aspect","BUIEff","CBH","CFL","ISI")
# Primary outputs:
system.time(foo<-fbpRaster(input = input))
# Using the "select" option:
system.time(foo<-fbpRaster(input = input,select=c("HFI","TFC", "ROS")))
# Secondary outputs:
system.time(foo<-fbpRaster(input = input,output="S"))
# All outputs:
#system.time(foo<-fbpRaster(input = input,output="A"))

### Additional, longer running examples  ###
# Keep only the required input layers, the other layers would be
# assigned with default values:
# keep only the required inputs:
dat0<-input[[c("FuelType","LAT","LONG","FFMC","BUI","WS","GS", "Dj","Aspect")]]
system.time(foo<-fbpRaster(input = dat0,output="A"))

\end{ExampleCode}
\end{Examples}
\inputencoding{utf8}
\HeaderA{fireSeason}{Fire Season Start and End}{fireSeason}
\keyword{methods}{fireSeason}
%
\begin{Description}\relax
\code{\LinkA{fireSeason}{fireSeason}} calculates the start and end fire season dates for
a given weather station. The current method used in the function is based on
three consecutive daily maximum temperature thresholds (Wotton and Flannigan
1993, Lawson and Armitage 2008). This function process input from a single
weather station.
\end{Description}
%
\begin{Usage}
\begin{verbatim}
fireSeason(
  input,
  fs.start = 12,
  fs.end = 5,
  method = "WF93",
  consistent.snow = FALSE,
  multi.year = FALSE
)
\end{verbatim}
\end{Usage}
%
\begin{Arguments}
\begin{ldescription}
\item[\code{input}] A data.frame containing input variables of including the
date/time and daily maximum temperature. Variable names have to be the same
as in the following list, but they are case insensitive. The order in which
the input variables are entered is not important either.


\Tabular{lll}{ 
\var{yr} & (required) & Year of the observations\\{}
\var{mon} & (required) & Month of the observations\\{} 
\var{day} & (required) & Day of the observations\\{} 
\var{tmax} & (required) & Maximum Daily Temperature (degrees C)\\{} 
\var{snow\_depth} & (optional) & Is consistent snow data in the input?\\{} }.

\item[\code{fs.start}] Temperature threshold (degrees C) to start the fire season
(default=12)

\item[\code{fs.end}] Temperature threshold (degrees C) to end the fire season
(default=5)

\item[\code{method}] Method of fire season calculation. Options are "wf93"" or
"la08" (default=WF93)

\item[\code{consistent.snow}] Is consistent snow data in the input? (default=FALSE)

\item[\code{multi.year}] Should the fire season span multiple years?
(default=FALSE)
\end{ldescription}
\end{Arguments}
%
\begin{Details}\relax
An important aspect to consider when calculating Fire Weather Index (FWI)
System variables is a definition of the fire season start and end dates
(Lawson and Armitage 2008). If a user starts calculations on a fire season
too late in the year, the FWI System variables may take too long to reach
equilibrium, thus throwing off the resulting indices. This function presents
two method of calculating these start and end dates, adapted from Wotton and
Flannigan (1993), and Lawson and Armitage (2008). The approach taken in this
function starts the fire season after three days of maximum temperature
greater than 12 degrees Celsius. The end of the fire season is determined
after three consecutive days of maximum temperature less than 5 degrees
Celsius.  The two temperature thresholds can be adjusted as parameters in
the function call. In regions where temperature thresholds will not end a
fire season, it is possible for the fire season to span multiple years, in
this case setting the multi.year parameter to TRUE will allow these
calculations to proceed.

This fire season length definition can also feed in to the overwinter DC
calculations (\LinkA{wDC}{wDC}). View the cffdrs package help files for an example
of using the \code{fireSeason}, \LinkA{wDC}{wDC}, and \LinkA{fwi}{fwi} functions in
conjunction.
\end{Details}
%
\begin{Value}
\LinkA{fireSeason}{fireSeason} returns a data frame of season and start and end
dates. Columns in data frame are described below.

Primary FBP output includes the following 8 variables: 
\begin{ldescription}
\item[\code{yr }] Year of the fire season start/end date
\item[\code{mon }] Month of the fire season start/end date
\item[\code{day }] Day of the fire season start/end date
\item[\code{fsdatetype }] Fire season date type (values are either "start" or "end")
\item[\code{date}] Full date value
\end{ldescription}
\end{Value}
%
\begin{Author}\relax
Alan Cantin, Xianli Wang, Mike Wotton, and Mike Flannigan
\end{Author}
%
\begin{References}\relax
Wotton, B.M. and Flannigan, M.D. (1993). Length of the fire
season in a changing climate. Forestry Chronicle, 69, 187-192.

\url{http://www.ualberta.ca/~flanniga/publications/1993_Wotton_Flannigan.pdf}

Lawson, B.D. and O.B. Armitage. 2008. Weather guide for the Canadian Forest
Fire Danger Rating System. Nat. Resour. Can., Can. For. Serv., North. For.
Cent., Edmonton, AB \url{http://cfs.nrcan.gc.ca/pubwarehouse/pdfs/29152.pdf}
\end{References}
%
\begin{SeeAlso}\relax
\code{\LinkA{fwi}{fwi}, \LinkA{wDC}{wDC}}
\end{SeeAlso}
%
\begin{Examples}
\begin{ExampleCode}

library(cffdrs)
#The standard test data:
data("test_wDC")
print(head(test_wDC))
## Sort the data:
input <- with(test_wDC, test_wDC[order(id,yr,mon,day),])

#Using the default fire season start and end temperature 
#thresholds:
a_fs <- fireSeason(input[input$id==1,])

#Check the result:
a_fs

#    yr mon day fsdatetype
#1 1999   5   4      start
#2 1999   5  12        end
#3 1999   5  18      start
#4 1999   5  25        end
#5 1999   5  30      start
#6 1999  10   6        end
#7 2000   6  27      start
#8 2000  10   7        end

#In the resulting data frame, the fire season starts 
#and ends multiple times in the first year. It is up to the user #for how to interpret this.

#modified fire season start and end temperature thresholds
a_fs <- fireSeason (input[input$id==1,],fs.start=10, fs.end=3)
a_fs
#    yr mon day fsdatetype
#1 1999   5   2      start
#2 1999  10  20        end
#3 2000   6  16      start
#4 2000  10   7        end
#select another id value, specify method explicitly
b_fs <- fireSeason(input[input$id==2,],method="WF93")
#print the calculated fireseason
b_fs
#   yr mon day fsdatetype
#1 1980   4  21      start
#2 1980   9  19        end
#3 1980  10   6      start
#4 1980  10  16        end
#5 1981   5  21      start
#6 1981  10  13        end

\end{ExampleCode}
\end{Examples}
\inputencoding{utf8}
\HeaderA{fwi}{Fire Weather Index System}{fwi}
\keyword{methods}{fwi}
%
\begin{Description}\relax
\code{fwi} is used to calculate the outputs of the Canadian Forest Fire
Weather Index (FWI) System for one day or one fire season based on noon
local standard time (LST) weather observations of temperature, relative
humidity, wind speed, and 24-hour rainfall, as well as the previous day's
fuel moisture conditions. This function could be used for either one weather
station or for multiple weather stations.
\end{Description}
%
\begin{Usage}
\begin{verbatim}
fwi(
  input,
  init = data.frame(ffmc = 85, dmc = 6, dc = 15, lat = 55),
  batch = TRUE,
  out = "all",
  lat.adjust = TRUE,
  uppercase = TRUE
)
\end{verbatim}
\end{Usage}
%
\begin{Arguments}
\begin{ldescription}
\item[\code{input}] A dataframe containing input variables of daily weather
observations taken at noon LST. Variable names have to be the same as in the
following list, but they are case insensitive. The order in which the input
variables are entered is not important.


\Tabular{lll}{ 
\var{id} & (optional) 
& Unique identifier of a weather\\{}
&& station or spatial point (no restriction on\\{}
&& data type); required when \code{batch=TRUE}\\{} 
\var{lat} & (recommended) & Latitude (decimal degree, default=55)\\{} 
\var{long} & (optional) & Longitude (decimal degree)\\{} 
\var{yr} & (optional) & Year of observation;
required when \code{batch=TRUE}\\{} 
\var{mon} & (recommended) & Month of the year (integer 1-12, default=7)\\{} 
\var{day} & (optional) & Dayof the month (integer); required when \code{batch=TRUE}\\{} 
\var{temp} & (required) & Temperature (centigrade)\\{} 
\var{rh} & (required) & Relative humidity (\%)\\{} 
\var{ws} & (required) & 10-m height wind speed (km/h)\\{} 
\var{prec} & (required) & 24-hour rainfall (mm)\\{} }

\item[\code{init}] A data.frame or vector contains either the initial values for
FFMC, DMC, and DC or the same variables that were calculated for the
previous day and will be used for the current day's calculation. The
function also accepts a vector if the initial or previous day FWI values is
for only one weather station (a warning message comes up if a single set of
initial values is used for multiple weather stations). Defaults are the
standard initial values for FFMC, DMC, and DC defined as the following:

\Tabular{ll}{ 
\var{ffmc} & Fine Fuel Moisture Code (FFMC; unitless) \\{}
& of the previous day.Default value is 85.\\{} 
\var{dmc} & Duff Moisture Code (DMC; unitless)\\{}
& of the previous day. Default value is 6.\\{} 
\var{dc} & Drought Code (DC; unitless)\\{}
& of the previous day. Default value is 15.\\{}
\var{lat} 
& Latitude of the weather station (optional, default=55).\\{}
& Latitude values are used to make\\{}
& day length adjustments in the function.\\{} }

\item[\code{batch}] Whether the computation is iterative or single step, default is
TRUE. When \code{batch=TRUE}, the function will calculate daily FWI System
outputs for one weather station over a period of time chronologically with
the initial conditions given (\code{init}) applied only to the first day of
calculation. If multiple weather stations are processed, an additional "id"
column is required in the input to label different stations, and the data
needs to be sorted by date/time and "id".  If \code{batch=FALSE}, the
function calculates only one time step (1 day) base on either the initial
start values or the previous day's FWI System variables, which should also
be assigned to \code{init} argument.

\item[\code{out}] The function offers two output options, \code{out="all"} will
produce a data frame that includes both the input and the FWI System
outputs; \code{out="fwi"} will generate a data frame with only the FWI
system components.

\item[\code{lat.adjust}] The function offers options for whether day length
adjustments should be applied to the calculations.  The default value is
"TRUE".

\item[\code{uppercase}] Output in upper cases or lower cases would be decided by
this argument. Default is TRUE.
\end{ldescription}
\end{Arguments}
%
\begin{Details}\relax
The Canadian Forest Fire Weather Index (FWI) System is a major subsystem of
the Canadian Forest Fire Danger Rating System, which also includes Canadian
Forest Fire Behavior Prediction (FBP) System. The modern FWI System was
first issued in 1970 and is the result of work by numerous researchers from
across Canada. It evolved from field research which began in the 1930's and
regional fire hazard and fire danger tables developed from that early
research.

The modern System (Van Wagner 1987) provides six output indices which
represent fuel moisture and potential fire behavior in a standard pine
forest fuel type. Inputs are a daily noon observation of fire weather, which
consists of screen-level air temperature and relative humidity, 10 meter
open wind speed and 24 accumulated precipitation.

The first three outputs of the system (the Fire Fuel Moisture Code (ffmc),
the Duff Moisture Code (dmc), and the Drought Code (dc)) track moisture in
different layers of the fuel making up the forest floor. Their calculation
relies on the daily fire weather observation and also, importantly, the
moisture code value from the previous day as they are in essence bookkeeping
systems tracking the amount of moisture (water) in to and out of the layer.
It is therefore important that when calculating FWI System outputs over an
entire fire season, an uninterrupted daily weather stream is provided; one
day is the assumed time step in the models and thus missing data must be
filled in.

The next three outputs of the System are relative (unitless) indicators of
aspects of fire behavior potential: spread rate (the Initial Spread Index,
isi), fuel consumption (the Build-up Index, bui) and fire intensity per unit
length of fire front (the Fire Weather Index, fwi).  This final index, the
fwi, is the component of the System used to establish the daily fire danger
level for a region and communicated to the public.  This final index can be
transformed to the Daily Severity Rating (dsr) to provide a more
reasonably-scaled estimate of fire control difficulty.

Both the Duff Moisture Code (dmc) and Drought Code (dc) are influenced by
day length (see Van Wagner 1987). Day length adjustments for different
ranges in latitude can be used (as described in Lawson and Armitage 2008
(\url{http://cfs.nrcan.gc.ca/pubwarehouse/pdfs/29152.pdf})) and are included
in this R function; latitude must be positive in the northern hemisphere and
negative in the southern hemisphere.

The default initial (i.e., "start-up") fuel moisture code values (FFMC=85,
DMC=6, DC=15) provide a reasonable set of conditions for most springtime
conditions in Canada, the Northern U.S., and Alaska. They are not suitable
for particularly dry winters and are presumably not appropriate for
different parts of the world.
\end{Details}
%
\begin{Value}
\code{fwi} returns a dataframe which includes both the input and the
FWI System variables as described below: \begin{ldescription}
\item[\code{Input Variables }] Including
temp, rh, ws, and prec with id, long, lat, yr, mon, or day as optional.
\item[\code{ffmc }] Fine Fuel Moisture Code\item[\code{dmc }] Duff Moisture Code
\item[\code{dc }] Drought Code\item[\code{isi }] Initial Spread Index\item[\code{bui
}] Buildup Index\item[\code{fwi }] Fire Weather Index\item[\code{dsr }] Daily Severity
Rating
\end{ldescription}
\end{Value}
%
\begin{Author}\relax
Xianli Wang, Alan Cantin, Marc-André Parisien, Mike Wotton, Kerry
Anderson, and Mike Flannigan
\end{Author}
%
\begin{References}\relax
1. Van Wagner, C.E. and T.L. Pickett. 1985. Equations and
FORTRAN program for the Canadian Forest Fire Weather Index System. Can. For.
Serv., Ottawa, Ont. For. Tech. Rep. 33. 18 p.
\url{http://cfs.nrcan.gc.ca/pubwarehouse/pdfs/19973.pdf}

2. Van Wagner, C.E. 1987. Development and structure of the Canadian forest
fire weather index system. Forest Technology Report 35. (Canadian Forestry
Service: Ottawa). \url{http://cfs.nrcan.gc.ca/pubwarehouse/pdfs/19927.pdf}

3.  Lawson, B.D. and O.B. Armitage. 2008. Weather guide for the Canadian
Forest Fire Danger Rating System. Nat. Resour. Can., Can. For. Serv., North.
For. Cent., Edmonton, AB.
\url{http://cfs.nrcan.gc.ca/pubwarehouse/pdfs/29152.pdf}
\end{References}
%
\begin{SeeAlso}\relax
\code{\LinkA{fbp}{fbp}}, \code{\LinkA{fwiRaster}{fwiRaster}}, \code{\LinkA{gfmc}{gfmc}},
\code{\LinkA{hffmc}{hffmc}}, \code{\LinkA{hffmcRaster}{hffmcRaster}}, \code{\LinkA{sdmc}{sdmc}},
\code{\LinkA{wDC}{wDC}}, \code{\LinkA{fireSeason}{fireSeason}}
\end{SeeAlso}
%
\begin{Examples}
\begin{ExampleCode}

# library(cffdrs)
# The test data is a standard test
# dataset for FWI system (Van Wagner and Pickett 1985) 
# data("test_fwi")
# Show the data, which is already sorted by time:
# head(test_fwi)
# long  lat	yr	mon	day	temp	rh	ws	prec
# -100	40	1985	4	  13	17	  42	25	0
# -100	40	1985	4	  14	20	  21	25	2.4
# -100	40	1985	4	  15	8.5	  40	17	0
# -100	40	1985	4	  16	6.5	  25	6	0
# -100	40	1985	4	  17	13	  34	24	0

## (1) FWI System variables for a single weather station:
# Using the default initial values and batch argument, 
# the function calculate FWI variables chronically:
fwi.out1<-fwi(test_fwi) 				
# Using a different set of initial values:
fwi.out2<-fwi(test_fwi,init=data.frame(ffmc=80, dmc=10,dc=16, lat=50))
# This could also be done as the following:
fwi.out2<-fwi(test_fwi,init=data.frame(80,10,6,50))
# Or:
fwi.out2<-fwi(test_fwi,init=c(80,10,6,50))
# Latitude could be ignored, and the default value (55) will 
# be used:
fwi.out2<-fwi(test_fwi,init=data.frame(80,10,6))

## (2) FWI for one or multiple stations in a single day:
# Change batch argument to FALSE, fwi calculates FWI 
# components based on previous day's fwi outputs:

fwi.out3<-fwi(test_fwi,init=fwi.out1,batch=FALSE)                 
# Using a suite of initials, assuming variables from fwi.out1
# are the initial values for different records. 
init_suite<-fwi.out1[,c("FFMC","DMC","DC","LAT")]
# Calculating FWI variables for one day but with multiple
# stations. Because the calculations is for one time step, 
# batch=FALSE:
fwi.out4<-fwi(test_fwi,init=init_suite,batch=FALSE)

## (3) FWI for multiple weather stations over a period of time: 
#Assuming there are 4 weather stations in the test dataset, and they are 
# ordered by day:
test_fwi$day<-rep(1:(nrow(test_fwi)/4),each=4)
test_fwi$id<-rep(1:4,length(unique(test_fwi$day)))
# Running the function with the same default initial inputs, will receive a 
# warning message, but that is fine: 
fwi(test_fwi)

## (4) Daylength adjustment:
# Change latitude values where the monthly daylength adjustments
are different from the standard ones
test_fwi$lat<-22
# With daylength adjustment
fwi(test_fwi)[1:3,]
# Without daylength adjustment
fwi(test_fwi,lat.adjust=FALSE)[1:3,]

\end{ExampleCode}
\end{Examples}
\inputencoding{utf8}
\HeaderA{fwiRaster}{Raster-based Fire Weather Index System}{fwiRaster}
\keyword{methods}{fwiRaster}
%
\begin{Description}\relax
\code{fwiRaster} is used to calculate the outputs of the Canadian Forest
Fire Weather Index (FWI) System for one day based on noon local standard
time (LST) weather observations of temperature, relative humidity, wind
speed, and 24-hour rainfall, as well as the previous day's fuel moisture
conditions. This function takes rasterized input and generates raster maps
as outputs.
\end{Description}
%
\begin{Usage}
\begin{verbatim}
fwiRaster(
  input,
  init = c(ffmc = 85, dmc = 6, dc = 15),
  mon = 7,
  out = "all",
  lat.adjust = TRUE,
  uppercase = TRUE
)
\end{verbatim}
\end{Usage}
%
\begin{Arguments}
\begin{ldescription}
\item[\code{input}] A stack or brick containing rasterized daily weather
observations taken at noon LST. Variable names have to be the same as in the
following list, but they are case insensitive. The order in which the inputs
are entered is not important.


\Tabular{lll}{ 
\var{lat} & (recommended) & Latitude (decimal degree,
default=55)\\{} 
\var{temp} & (required) & Temperature (centigrade)\\{}
\var{rh} & (required) & Relative humidity (\%)\\{} 
\var{ws} &
(required) & 10-m height wind speed (km/h)\\{} 
\var{prec} & (required)
& 24-hour rainfall (mm)\\{} }

\item[\code{init}] A vector that contains the initial values for FFMC, DMC, and DC
or a stack that contains raster maps of the three moisture codes calculated
for the previous day, which will be used for the current day's calculation.
Defaults are the standard initial values for FFMC, DMC, and DC defined as
the following: 

\Tabular{ll}{ 
\var{ffmc} & Fine Fuel Moisture Code (FFMC; unitless) of the previous 
day. Default value is 85.\\{} 
\var{dmc} & Duff Moisture Code (DMC; unitless) of the previous day. 
Default value is 6.\\{}
\var{dc} & Drought Code (DC; unitless) of the previous day. Default value
is 15.\\{} 
\var{lat} & Latitude of the weather station (optional, default=55). 
Latitude values are used to make\\{}& day length adjustments
in the function.\\{} }

\item[\code{mon}] Month of the year (integer 1\textasciitilde{}12, default=7). Month is used in
latitude adjustment (\code{lat.adjust}), it is therefore recommended when
\code{lat.adjust=TRUE} was chosen.

\item[\code{out}] The function offers two output options, \code{out="all"} will
produce a raster stack include both the input and the FWI System outputs;
\code{out="fwi"} will generate a stack with only the FWI system components.

\item[\code{lat.adjust}] The function offers options for whether latitude
adjustments to day lengths should be applied to the calculations. The
default value is "TRUE".

\item[\code{uppercase}] Output in upper cases or lower cases would be decided by
this argument. Default is TRUE.
\end{ldescription}
\end{Arguments}
%
\begin{Details}\relax
The Canadian Forest Fire Weather Index (FWI) System is a major subsystem of
the Canadian Forest Fire Danger Rating System, which also includes Canadian
Forest Fire Behavior Prediction (FBP) System. The modern FWI System was
first issued in 1970 and is the result of work by numerous researchers from
across Canada. It evolved from field research which began in the 1930's and
regional fire hazard and fire danger tables developed from that early
research.

The modern System (Van Wagner 1987) provides six output indices which
represent fuel moisture and potential fire behavior in a standard pine
forest fuel type. Inputs are a daily noon observation of fire weather, which
consists of screen-level air temperature and relative humidity, 10 meter
open wind speed and 24 accumulated precipitation.

The first three outputs of the system (the Fire Fuel Moisture Code, the Duff
Moisture Code, and the Drought Code) track moisture in different layers of
the fuel making up the forest floor. Their calculation relies on the daily
fire weather observation and also, importantly, the code value from the
previous day as they are in essence bookkeeping systems tracking the amount
of moisture (water) in to and out of the layer.  It is therefore important
that when calculating FWI System outputs over an entire fire season, an
uninterrupted daily weather stream is provided; one day is the assumed time
step in the models and thus missing data must be filled in.

The next three outputs of the System are relative (unitless) indicators of
aspects of fire behavior potential: spread rate (the Initial Spread Index),
fuel consumption (the Build-up Index) and fire intensity per unit length of
fire front (the Fire Weather Index).  This final index, the fwi, is the
component of the System used to establish the daily fire danger level for a
region and communicated to the public.  This final index can be transformed
to the Daily Severity Rating (dsr) to provide a more reasonably-scaled
estimate of fire control difficulty.

Both the Duff Moisture Code (dmc) and Drought Code (dc) are influenced by
day length (see Van Wagner, 1987). Day length adjustments for different
ranges in latitude can be used (as described in Lawson and Armitage 2008
(\url{http://cfs.nrcan.gc.ca/pubwarehouse/pdfs/29152.pdf})) and are included
in this R function; latitude must be positive in the northern hemisphere and
negative in the southern hemisphere.

The default initial (i.e., "start-up") fuel moisture code values (FFMC=85,
DMC=6, DC=15) provide a reasonable set of conditions for most springtime
conditions in Canada, the Northern U.S., and Alaska. They are not suitable
for particularly dry winters and are presumably not appropriate for
different parts of the world.
\end{Details}
%
\begin{Value}
By default, \code{fwi} returns a raster stack which includes both
the input and the FWI System variables, as describe below: \begin{ldescription}
\item[\code{Inputs
}] Including \code{temp}, \code{rh}, \code{ws}, and \code{prec} with
\code{lat} as optional.\item[\code{ffmc }] Fine Fuel Moisture Code\item[\code{dmc
}] Duff Moisture Code\item[\code{dc }] Drought Code\item[\code{isi }] Initial Spread
Index\item[\code{bui }] Buildup Index\item[\code{fwi }] Fire Weather Index\item[\code{dsr
}] Daily Severity Rating
\end{ldescription}
\end{Value}
%
\begin{Author}\relax
Xianli Wang, Alan Cantin, Marc-André Parisien, Mike Wotton, Kerry
Anderson, and Mike Flannigan
\end{Author}
%
\begin{References}\relax
1. Van Wagner, C.E. and T.L. Pickett. 1985. Equations and
FORTRAN program for the Canadian Forest Fire Weather Index System. Can. For.
Serv., Ottawa, Ont. For. Tech. Rep. 33. 18 p.
\url{http://cfs.nrcan.gc.ca/pubwarehouse/pdfs/19973.pdf}

2. Van Wagner, C.E. 1987. Development and structure of the Canadian forest
fire weather index system. Forest Technology Report 35. (Canadian Forestry
Service: Ottawa). \url{http://cfs.nrcan.gc.ca/pubwarehouse/pdfs/19927.pdf}

3.  Lawson, B.D. and O.B. Armitage. 2008. Weather guide for the Canadian
Forest Fire Danger Rating System. Nat. Resour. Can., Can. For. Serv., North.
For. Cent., Edmonton, AB.
\url{http://cfs.nrcan.gc.ca/pubwarehouse/pdfs/29152.pdf}
\end{References}
%
\begin{SeeAlso}\relax
\code{\LinkA{fbp}{fbp}}, \code{\LinkA{fbpRaster}{fbpRaster}}, \code{\LinkA{fwi}{fwi}},
\code{\LinkA{hffmc}{hffmc}}, \code{\LinkA{hffmcRaster}{hffmcRaster}}
\end{SeeAlso}
%
\begin{Examples}
\begin{ExampleCode}

library(cffdrs)
require(raster)
# The test data is a stack with four input variables including 
# daily noon temp, rh, ws, and prec (we recommend tif format):
day01src <- system.file("extdata","test_rast_day01.tif",package="cffdrs")
day01 <- stack(day01src)
day01 <- crop(day01,c(250,255,47,51))
# assign variable names:
names(day01)<-c("temp","rh","ws","prec")
# (1) use the initial values
foo<-fwiRaster(day01)
plot(foo)
### Additional, longer running examples ###
# (2) use initial values with larger raster
day01 <- stack(day01src)
names(day01)<-c("temp","rh","ws","prec")
foo<-fwiRaster(day01)
plot(foo)

\end{ExampleCode}
\end{Examples}
\inputencoding{utf8}
\HeaderA{gfmc}{Grass Fuel Moisture Code}{gfmc}
\keyword{methods}{gfmc}
%
\begin{Description}\relax
\code{gfmc} calculates both the moisture content of the surface of a fully
cured matted grass layer and also an equivalent Grass Fuel Moisture Code
(gfmc) (Wotton, 2009) to create a parallel with the hourly ffmc (see the
\code{\LinkA{fwi}{fwi}} and \code{\LinkA{hffmc}{hffmc}}functions). The calculation is
based on hourly (or sub-hourly) weather observations of temperature,
relative humidity, wind speed, rainfall, and solar radiation. The user must
also estimate an initial value of the gfmc for the layer. This function
could be used for either one weather station or multiple weather stations.
\end{Description}
%
\begin{Usage}
\begin{verbatim}
gfmc(
  input,
  GFMCold = 85,
  batch = TRUE,
  time.step = 1,
  roFL = 0.3,
  out = "GFMCandMC"
)
\end{verbatim}
\end{Usage}
%
\begin{Arguments}
\begin{ldescription}
\item[\code{input}] A dataframe containing input variables of daily noon weather
observations. Variable names have to be the same as in the following list,
but they are case insensitive. The order in which the input variables are
entered is not important.


\Tabular{lll}{ 
\var{temp} & (required) & Temperature (centigrade)\\{}
\var{rh} & (required) & Relative humidity (\%)\\{} 
\var{ws} & (required) & 10-m height wind speed (km/h)\\{} 
\var{prec} & (required) & 1-hour rainfall (mm)\\{}
\var{isol} & (required) & Solar radiation (kW/m\textasciicircum{}2)\\{} 
\var{mon} & (recommended) & Month of the year (integer' 1-12)\\{} 
\var{day} & (optional) & Day of the month (integer)\\{} }

\item[\code{GFMCold}] Previous value of GFMC (i.e. value calculated at the previous
time step)[default is 85 (which corresponds to a moisture content of about
16\%)]. On the first calculation this is the estimate of the GFMC value at
the start of the time step. The \code{GFMCold} argument can accept a single
initial value for multiple weather stations, and also accept a vector of
initial values for multiple weather stations.  NOTE: this input represents
the CODE value, not a direct moisture content value. The CODE values in the
Canadian FWI System increase within decreasing moisture content. To roughly
convert a moisture content value to a CODE value on the FF-scale (used in
the FWI Systems FFMC) use \code{GFMCold} =101-gmc (where gmc is moisture
content in \%)

\item[\code{batch}] Whether the computation is iterative or single step, default is
TRUE. When \code{batch=TRUE}, the function will calculate hourly or
sub-hourly GFMC for one weather station over a period of time iteratively.
If multiple weather stations are processed, an additional "id" column is
required in the input to label different stations, and the data needs to be
sorted by time sequence and "id".  If \code{batch=FALSE}, the function
calculates only one time step (1 hour) base on either the previous hourly
GFMC or the initial start value.

\item[\code{time.step}] Time step (hour) [default 1 hour]

\item[\code{roFL}] The nominal fuel load of the fine fuel layer, default is 0.3
kg/m\textasciicircum{}2

\item[\code{out}] Output format, default is "GFMCandMC", which contains both GFMC
and moisture content (MC) in a data.frame format. Other choices include:
"GFMC", "MC", and "ALL", which include both the input and GFMC and MC.
\end{ldescription}
\end{Arguments}
%
\begin{Details}\relax
The Canadian Forest Fire Danger Rating System (CFFDRS) is used throughout
Canada, and in a number of countries throughout the world, for estimating
fire potential in wildland fuels. This new Grass Fuel Moisture Code (GFMC)
is an addition (Wotton 2009) to the CFFDRS and retains the structure of that
System's hourly Fine Fuel Moisture Code (HFFMC) (Van Wagner 1977). It tracks
moisture content in the top 5 cm of a fully-cured and fully-matted layer of
grass and thus is representative of typical after winter conditions in areas
that receive snowfall.  This new moisture calculation method outputs both
the actual moisture content of the layer and also the transformed moisture
Code value using the FFMC's FF-scale.  In the CFFDRS the moisture codes are
in fact relatively simple transformations of actual moisture content such
that decreasing moisture content (increasing dryness) is indicated by an
increasing Code value. This moisture calculation uses the same input weather
observations as the hourly FFMC, but also requires an estimate of solar
radiation incident on the fuel.
\end{Details}
%
\begin{Value}
\code{gfmc} returns GFMC and moisture content (MC) values
collectively (default) or separately.
\end{Value}
%
\begin{Author}\relax
Xianli Wang, Mike Wotton, Alan Cantin, and Mike Flannigan
\end{Author}
%
\begin{References}\relax
Wotton, B.M. 2009. A grass moisture model for the Canadian
Forest Fire Danger Rating System. In: Proceedings 8th Fire and Forest
Meteorology Symposium, Kalispell, MT Oct 13-15, 2009. Paper 3-2.
\url{https://ams.confex.com/ams/pdfpapers/155930.pdf}

Van Wagner, C.E. 1977. A method of computing fine fuel moisture content
throughout the diurnal cycle. Environment Canada, Canadian Forestry Service,
Petawawa Forest Experiment Station, Chalk River, Ontario. Information Report
PS-X-69. \url{http://cfs.nrcan.gc.ca/pubwarehouse/pdfs/25591.pdf}
\end{References}
%
\begin{SeeAlso}\relax
\code{\LinkA{fwi}{fwi}}, \code{\LinkA{hffmc}{hffmc}}
\end{SeeAlso}
%
\begin{Examples}
\begin{ExampleCode}

library(cffdrs)
#load the test data
data("test_gfmc")
# show the data format:
head(test_gfmc)
#     yr mon day hr temp   rh   ws prec  isol
# 1 2006   5  17 10 15.8 54.6  5.0    0 0.340
# 2 2006   5  17 11 16.3 52.9  5.0    0 0.380
# 3 2006   5  17 12 18.8 45.1  5.0    0 0.626
# 4 2006   5  17 13 20.4 40.8  9.5    0 0.656
# 5 2006   5  17 14 20.1 41.7  8.7    0 0.657
# 6 2006   5  17 15 18.6 45.8 13.5    0 0.629
# (1) gfmc default: 
# Re-order the data by year, month, day, and hour:
dat<-test_gfmc[with(test_gfmc,order(yr,mon,day,hr)),]
# Because the test data has 24 hours input variables 
# it is possible to calculate the hourly GFMC continuously 
# through multiple days(with the default initial GFMCold=85):
dat$gfmc_default<-gfmc(dat) 
# two variables will be added to the input, GFMC and MC
head(dat)
# (2) For multiple weather stations:
# One time step (1 hour) with default initial value:
  foo<-gfmc(dat,batch=FALSE)
# Chronical hourly GFMC with only one initial 
# value (GFMCold=85), but multiple weather stations. 
# Note: data is ordered by date/time and the station id. Subset 
# the data by keeping only the first 10 hours of observations 
# each day:
dat1<-subset(dat,hr%in%c(0:9))
#assuming observations were from the same day but with 
#9 different weather stations:
dat1$day<-NULL
dat1<-dat1[with(dat1,order(yr,mon,hr)),]
dat1$id<-rep(1:8,nrow(dat1)/8)
#check the data:
head(dat1)
# Calculate GFMC for multiple stations:
dat1$gfmc01<-gfmc(dat1,batch=TRUE)
# We can provide multiple initial GFMC (GFMCold) as a vector:   
dat1$gfmc02<- gfmc(dat1,GFMCold = sample(70:100,8, replace=TRUE),batch=TRUE)
# (3)output argument
## include all inputs and outputs:
dat0<-dat[with(dat,order(yr,mon,day,hr)),]
foo<-gfmc(dat,out="ALL")
## subhourly time step:
gfmc(dat0,time.step=1.5)

\end{ExampleCode}
\end{Examples}
\inputencoding{utf8}
\HeaderA{gfmcRaster}{Grass Fuel Moisture Raster Calculation}{gfmcRaster}
%
\begin{Description}\relax
Calculation of the Grass Fuel Moisture Code. This calculates the
moisture content of both the surface of a fully cured matted grass layer and 
also an equivalent Grass Fuel Moisture Code. All equations come from Wotton 
(2009) as cited below unless otherwise specified.
\end{Description}
%
\begin{Usage}
\begin{verbatim}
gfmcRaster(input, GFMCold = 85, time.step = 1, roFL = 0.3, out = "GFMCandMC")
\end{verbatim}
\end{Usage}
%
\begin{Arguments}
\begin{ldescription}
\item[\code{input}] (raster stack)

\Tabular{lll}{
\var{temp} & (required) & Temperature (centigrade)\\{}
\var{rh}   & (required) & Relative humidity (\%)\\{} 
\var{ws}   & (required) & 10-m height wind speed (km/h)\\{} 
\var{prec} & (required) & 1-hour rainfall (mm)\\{}
\var{isol} & (required) & Solar radiation (kW/m\textasciicircum{}2)\\{} }

\item[\code{GFMCold}] GFMC from yesterday (double, default=85)

\item[\code{time.step}] The hourly time steps (integer hour, default=1)

\item[\code{roFL}] Nominal fuel load of the fine fuel layer (kg/m\textasciicircum{}2 double, default=0.3)

\item[\code{out}] Output format (GFMCandMC/MC/GFMC/ALL, default=GFMCandMC)
\end{ldescription}
\end{Arguments}
%
\begin{Value}
Returns a raster stack of either MC, GMFC, All, or GFMC and MC
\end{Value}
%
\begin{References}\relax
Wotton, B.M. 2009. A grass moisture model for the Canadian
Forest Fire Danger Rating System. In: Proceedings 8th Fire and
Forest Meteorology Symposium, Kalispell, MT Oct 13-15, 2009.
Paper 3-2. \url{https://ams.confex.com/ams/pdfpapers/155930.pdf}
\end{References}
\inputencoding{utf8}
\HeaderA{hffmc}{Hourly Fine Fuel Moisture Code}{hffmc}
\keyword{methods}{hffmc}
%
\begin{Description}\relax
\code{hffmc} is used to calculate hourly Fine Fuel Moisture Code (FFMC) and
is based on a calculation routine first described in detail by Van Wagner
(1977) and which has been updated in minor ways by the Canadian Forest
Service to have it agree with the calculation methodology for the daily FFMC
(see \code{\LinkA{fwi}{fwi}}).  In its simplest typical use this current routine
calculates a value of FFMC based on a series of uninterrupted hourly weather
observations of screen level (\textasciitilde{}1.4 m) temperature, relative humidity, 10 m
wind speed, and 1-hour rainfall. This implementation of the function
includes an optional time.step input which is defaulted to one hour, but can
be reduced if sub-hourly calculation of the code is needed.  The FFMC is in
essence a bookkeeping system for moisture content and thus it needs to use
the last time.step's value of FFMC in its calculation as well.  This
function could be used for either one weather station or for multiple
weather stations.
\end{Description}
%
\begin{Usage}
\begin{verbatim}
hffmc(
  weatherstream,
  ffmc_old = 85,
  time.step = 1,
  calc.step = FALSE,
  batch = TRUE,
  hourlyFWI = FALSE
)
\end{verbatim}
\end{Usage}
%
\begin{Arguments}
\begin{ldescription}
\item[\code{weatherstream}] A dataframe containing input variables of hourly
weather observations. It is important that variable names have to be the
same as in the following list, but they are case insensitive. The order in
which the input variables are entered is not important.


\Tabular{lll}{ 
\var{temp} & (required) & Temperature (centigrade)\\{}
\var{rh} & (required) & Relative humidity (\%)\\{} 
\var{ws} & (required) & 10-m height wind speed (km/h)\\{} 
\var{prec} & (required) & 1-hour rainfall (mm)\\{} 
\var{hr} & (optional) & Hourly value to calculate sub-hourly ffmc \\{} 
\var{bui} & (optional) & Daily BUI value for the computation of hourly 
FWI. It is required when \code{hourlyFWI=TRUE}.\\{} } 
Typically this dataframe also contains date and
hour fields so outputs can be associated with a specific day and time,
however these fields are not used in the calculations.  If multiple weather
stations are being used, a weather station ID field is typically included as
well, though this is simply for bookkeeping purposes and does not affect the
calculation.

\item[\code{ffmc\_old}] Initial FFMC. At the start of calculations at a particular
station there is a need to provide an estimate of the FFMC in the previous
timestep; this is because the FFMC is, in essence, a bookkeeping system for
moisture.  If no estimate of previous hour's FFMC is available the function
will use default value, \code{ffmc\_old=85}. When using the routine to
calculate hourly FFMC at multiple stations the \code{ffmc\_old} argument can
also accept a vector with the same number of weather stations.

\item[\code{time.step}] Is the time (in hours) between the previous value of FFMC
and the current time at which we want to calculate a new value of the FFMC.
When not specified it will take on a default value of \code{time.step=1}.

\item[\code{calc.step}] Optional for whether time step between two observations is
calculated. Default is FALSE, no calculations. This is used when time
intervals are not uniform in the input.

\item[\code{batch}] Whether the computation is iterative or single step, default is
TRUE. When \code{batch=TRUE}, the function will calculate hourly or
sub-hourly FFMC for one weather station over a period of time iteratively.
If multiple weather stations are processed, an additional "id" column is
required in the input weatherstream to label different stations, and the
data needs to be sorted by date/time and "id".  If \code{batch=FALSE}, the
function calculates only one time step base on either the previous hourly
FFMC or the initial start value.

\item[\code{hourlyFWI}] Optional for the computation of hourly ISI, FWI, and DSR.
Default is FALSE. While \code{hourlyFWI=TRUE}, daily BUI is required for the
computation of FWI.
\end{ldescription}
\end{Arguments}
%
\begin{Details}\relax
The hourly FFMC is very similar in its structure and calculation to the
Canadian Forest Fire Weather Index System's daily FFMC (\code{\LinkA{fwi}{fwi}})
but has an altered drying and wetting rate which more realistically reflects
the drying and wetting of a pine needle litter layer sitting on a decaying
organic layer.  This particular implementation of the Canadian Forest Fire
Danger Rating System's hourly FFMC provides for a flexible time step; that
is, the data need not necessarily be in time increments of one hour.  This
flexibility has been added for some users who use this method with data
sampled more frequently that one hour.  We do not recommend using a time
step much greater than one hour. An important and implicit assumption in
this calculation is that the input weather is constant over the time step of
each calculation (e.g., typically over the previous hour).  This is a
reasonable assumption for an hour; however it can become problematic for
longer periods.  For brevity we have referred to this routine throughout
this description as the hourly FFMC.

Because of the shortened time step, which can lead to more frequent
calculations and conversion between moisture content and the code value
itself, we have increased the precision of one of the constants in the
simple formula that converts litter moisture content to the 'Code' value.
This is necessary to avoid a potential bias that gets introduced during
extremely dry conditions.  This is simply a change in the precision at which
this constant is used in the equation and is not a change to the standard
FFMC conversion between moisture and code value (which is referred to as the
FF-scale).

The calculation requires the previous hour's FFMC as an input to the
calculation of the current hour's FFMC; this is because the routine can be
thought of as a bookkeeping system and needs to know the amount of moisture
being held in the fuel prior to any drying or wetting in the current period.
After each hour's calculation that newly calculated FFMC simply becomes the
starting FFMC in the next hour's calculation.  At the beginning of the
calculations at a station this previous hours FFMC must be estimated. It is
typical to use a value of 85 when this value cannot be estimated more
accurately; this code value corresponds to a moisture content of about 16\%
in typical pine litter fuels.
\end{Details}
%
\begin{Value}
\code{hffmc} returns a vector of hourly or sub-hourly FFMC values,
which may contain 1 or multiple elements. Optionally when
\code{hourlyFWI=TRUE}, the function also output a data.frame contains input
weatherstream as well as the hourly or sub-hourly FFMC, ISI, FWI, and DSR.
\end{Value}
%
\begin{Author}\relax
Xianli Wang, Mike Wotton, Alan Cantin, Brett Moore, and Mike
Flannigan
\end{Author}
%
\begin{References}\relax
Van Wagner, C.E. 1977. A method of computing fine fuel moisture
content throughout the diurnal cycle. Environment Canada, Canadian Forestry
Service, Petawawa Forest Experiment Station, Chalk River, Ontario.
Information Report PS-X-69.
\url{http://cfs.nrcan.gc.ca/pubwarehouse/pdfs/25591.pdf}
\end{References}
%
\begin{SeeAlso}\relax
\code{\LinkA{fbp}{fbp}}, \code{\LinkA{fwi}{fwi}}, \code{\LinkA{hffmcRaster}{hffmcRaster}}
\end{SeeAlso}
%
\begin{Examples}
\begin{ExampleCode}

library(cffdrs)
data("test_hffmc")
# show the data format:
head(test_hffmc)
# (1)hffmc default: 
# Re-order the data by year, month, day, and hour:
test_hffmc<-test_hffmc[with(test_hffmc, order(yr,mon,day,hr)),]
# Because the test data has 24 hours input variables 
# it is possible to calculate the hourly FFMC chronically 
# through multiple days(with the default initial ffmc_old=85):
test_hffmc$ffmc_default<-hffmc(test_hffmc) 
# (2) Calculate FFMC for multiple stations:
# Calculate hourly FFMC with only one initial 
# value (ffmc_old=85), but multiple weather stations. 
# Sort the input by date/time and the station id:
test_hffmc<-test_hffmc[with(test_hffmc,order(yr,mon,hr)),]
# Add weather station id:
test_hffmc$id<-rep(1:10,nrow(test_hffmc)/10)
#check the data:
head(test_hffmc)
test_hffmc$ffmc01<-hffmc(test_hffmc,batch=TRUE)
# With multiple initial FFMC (ffmc_old) as a vector: 
test_hffmc$ffmc02<- hffmc(test_hffmc,ffmc_old = sample(70:100,10, replace=TRUE),batch=TRUE)
# One time step assuming all records are from different 
# weather stations: 
     foo<-hffmc(test_hffmc,batch=FALSE)
# (3) output all hourly FWI System variables:
test_hffmc$id<-NULL
test_hffmc<-test_hffmc[with(test_hffmc,    order(yr,mon,day,hr)),]
foo<-hffmc(test_hffmc,hourlyFWI=TRUE)
# this will not run: warning message requesting for daily BUI
test_hffmc$bui<-100
foo<-hffmc(test_hffmc,hourlyFWI=TRUE)
# (4) Calculate time steps in case the time intervals are 
# not uniform:
dat0<-test_hffmc[sample(1:30,20),]
dat0<-dat0[with(dat0,order(yr,mon,day,hr)),]
# with or without calc.step, hffmc is going to generate
# different FFMC values.
# without calculating time step (default):
hffmc(dat0,time.step=1)
# with calc.step=TRUE, time.step=1 is applied to 
# only the first record, the rests would be calculated:
hffmc(dat0,time.step=1,calc.step=TRUE)

\end{ExampleCode}
\end{Examples}
\inputencoding{utf8}
\HeaderA{hffmcRaster}{Raster-based Hourly Fine Fuel Moisture Code}{hffmcRaster}
\keyword{methods}{hffmcRaster}
%
\begin{Description}\relax
\code{hffmcRaster} is used to calculate hourly Fine Fuel Moisture Code
(FFMC) based on hourly weather observations of screen level (\textasciitilde{}1.4 m)
temperature, relative humidity, 10 m wind speed, and 1-hour rainfall. This
implementation of the function includes an optional timestep input which is
defaulted to one hour, but can be reduced if sub-hourly calculation of the
code is needed.  The FFMC is in essence a bookkeeping system for moisture
content and thus it needs to use the last timestep's value of FFMC in its
calculation. \code{hffmcRaster} takes rasterized inputs and generates raster
maps as outputs.
\end{Description}
%
\begin{Usage}
\begin{verbatim}
hffmcRaster(weatherstream, ffmc_old = 85, time.step = 1, hourlyFWI = FALSE)
\end{verbatim}
\end{Usage}
%
\begin{Arguments}
\begin{ldescription}
\item[\code{weatherstream}] A stack or brick containing rasterized hourly weather
observations. Variable names have to be the same as in the following list,
but they are case insensitive. The order in which the input variables are
entered is not required.


\Tabular{lll}{ 
\var{temp} & (required) & Temperature (centigrade)\\{}
\var{rh} & (required) & Relative humidity (\%)\\{} 
\var{ws} & (required) & 10-m height wind speed (km/h)\\{} 
\var{prec} & (required) & 1-hour rainfall (mm)\\{} 
\var{bui} & (optional) & Daily BUI value for the computation of hourly
FWI. It is required when \code{hourlyFWI=TRUE}.\\{} }

\item[\code{ffmc\_old}] A single value of FFMC or a raster of FFMC for the previous
hour which will be used for the current hour's calculation. In some
situations, there are no previous-hourly FFMC values to calculate the
current hourly FFMC, the function will use a default value,
\code{ffmc\_old=84}.

\item[\code{time.step}] timestep in hours. Default is 1 hour, set for standard
hourly FFMC calculation. While \code{time.step} is set to values with
decimal places, sub-hourly FFMC would be calculated.

\item[\code{hourlyFWI}] Optional for the computation of hourly ISI, FWI, and DSR.
Default is FALSE. While \code{hourlyFWI=TRUE}, daily BUI is required for the
computation of FWI.
\end{ldescription}
\end{Arguments}
%
\begin{Details}\relax
The hourly FFMC is very similar in its structure and calculation to the
Canadian Forest Fire Weather Index System's daily FFMC (\code{\LinkA{fwi}{fwi}})
but has an altered drying and wetting rate which more realistically reflects
the drying and wetting of a pine needle litter layer sitting on a decaying
organic layer.  This particular implementation of the Canadian Forest Fire
Danger Rating System's hourly FFMC provides for a flexible timestep; that
is, the data need not necessarily be in time increments of one hour.  This
flexibility has been added for some users who use this method with data
sampled more frequently that one hour.  We do not recommend using a timestep
much greater than one hour. An important and implicit assumption in this
calculation is that the input weather is constant over the timestep of each
calculation (e.g., typically over the previous hour).  This is a reasonable
assumption for an hour; however it can become problematic for longer
periods.  For brevity we have referred to this routine throughout this
description as the hourly FFMC.

Because of the shortened timestep, which can lead to more frequent
calculations and conversion between moisture content and the code value
itself, we have increased the precision of one of the constants in the
simple formula that converts litter moisture content to the 'Code' value.
This is necessary to avoid a potential bias that gets introduced during
extremely dry conditions.  This is simply a change in the precision at which
this constant is used in the equation and is not a change to the standard
FFMC conversion between moisture and code value (which is referred to as the
FF-scale).

The calculation requires the previous hour's FFMC as an input to the
calculation of the current hour's FFMC; this is because the routine can be
thought of as a bookkeeping system and needs to know the amount of moisture
being held in the fuel prior to any drying or wetting in the current period.
After each hour's calculation that newly calculated FFMC simply becomes the
starting FFMC in the next hour's calculation.  At the beginning of the
calculations at a station this previous hours FFMC must be estimated. It is
typical to use a value of 85 when this value cannot be estimated more
accurately; this code value corresponds to a moisture content of about 16\%
in typical pine litter fuels.
\end{Details}
%
\begin{Value}
\code{hffmcRaster} returns a vector of hourly or sub-hourly FFMC
values, which may contain 1 or multiple elements. Optionally when
\code{hourlyFWI=TRUE}, the function also output a data.frame contains input
weatherstream as well as the hourly or sub-hourly FFMC, ISI, FWI, and DSR.
\end{Value}
%
\begin{Author}\relax
Xianli Wang, Mike Wotton, Alan Cantin, Brett Moore, and Mike
Flannigan
\end{Author}
%
\begin{References}\relax
Van Wagner, C.E. 1977. A method of computing fine fuel moisture
content throughout the diurnal cycle. Environment Canada, Canadian Forestry
Service, Petawawa Forest Experiment Station, Chalk River, Ontario.
Information Report PS-X-69.
\url{http://cfs.nrcan.gc.ca/pubwarehouse/pdfs/25591.pdf}
\end{References}
%
\begin{SeeAlso}\relax
\code{\LinkA{fbp}{fbp}}, \code{\LinkA{fwi}{fwi}}, \code{\LinkA{hffmc}{hffmc}}
\end{SeeAlso}
%
\begin{Examples}
\begin{ExampleCode}

library(cffdrs)
require(raster)
## load the test data for the first hour, namely hour01:
hour01src <- system.file("extdata","test_rast_hour01.tif",package="cffdrs")
hour01 <- stack(hour01src)
# Assign names to the layers:
names(hour01)<-c("temp","rh","ws","prec")
# (1) Default, based on the initial value: 
foo<-hffmcRaster(hour01)
plot(foo)
### Additional, longer running examples ###
# (2) Based on previous day's hffmc:
# load the test data for the second hour, namely hour02:
hour02src <- system.file("extdata","test_rast_hour02.tif",package="cffdrs")
hour02 <- stack(hour02src)
# Assign variable names to the layers:
names(hour02)<-c("temp","rh","ws","prec")
foo1<-hffmcRaster(hour02,ffmc_old=foo)
plot(foo1)
# (3) Calculate other hourly FWI components (ISI, FWI, and DSR):
# Need BUI layer, 
bui<-hour02$temp
values(bui)<-50
hour02<-stack(hour02,bui)
# Re-assign variable names to the layers:
names(hour02)<-c("temp","rh","ws","prec","bui")
# Calculate all the variables:
foo2<-hffmcRaster(hour02,ffmc_old=foo,hourlyFWI=TRUE)
# Visualize the maps:
plot(foo2)

\end{ExampleCode}
\end{Examples}
\inputencoding{utf8}
\HeaderA{lros}{Line-based input for Simard Rate of Spread and Direction}{lros}
\keyword{ros}{lros}
%
\begin{Description}\relax
\code{lros} is used to calculate the rate of spread and direction given one
set of three point-based observations of fire arrival time. The function
requires that the user specify the time that the fire crossed each point,
along with the measured lengths between each pair of observational points,
and a reference bearing (one specified side of the triangle). This function
allows quick input of a dataframe specifying one or many triangles.
\end{Description}
%
\begin{Usage}
\begin{verbatim}
lros(input)
\end{verbatim}
\end{Usage}
%
\begin{Arguments}
\begin{ldescription}
\item[\code{input}] A dataframe containing input variables of time fire front
crossed points 1, 2, 3, and latitude/longitude for those same points.
Variable names have to be the same as in the following list, but they are
case insensitive. The order in which the input variables are entered is not
important.


\Tabular{lll}{ 
\var{T1} & (required) & Time that the fire front crossed point 1. Time 
entered in fractional \\{}&& format. Output ROS will depend on the level
of precision entered \\{}&& (minute, second, decisecond)\\{} 
\var{T2} & (required) & Time that the fire front crossed point 2. Time 
entered in fractional \\{}&& format. Output ROS will depend on the level
of precision entered \\{}&& (minute, second, decisecond)\\{} 
\var{T3} & (required) & Time that the fire front crossed point 3. Time 
entered in fractional \\{}&& format. Output ROS will depend on the level 
of precision entered \\{}&& (minute, second, decisecond)\\{} 
\var{LengthT1T2}& (required) & Length between each pair of observation 
points T1 and T2 (subscripts \\{}&& denote time-ordered pairs). (meters)\\{} 
\var{LengthT2T3}& (required) & Length between each pair of observation 
points T2 and T3 (subscripts \\{}&& denote time-ordered pairs). (meters)\\{} 
\var{LengthT1T3}& (required) & Length between each pair of observation 
points T1 and T3 (subscripts \\{}&& denote time-ordered pairs). (meters)\\{} 
\var{BearingT1T2} & (required) & Reference bearing. For reference, 
North = 0, West = -90, East = 90 (degrees)\\{} 
\var{BearingT1T3} & (required) & Reference bearing. For reference, 
North = 0, West = -90, East = 90 (degrees)\\{} }
\end{ldescription}
\end{Arguments}
%
\begin{Details}\relax
\code{lros} Allows R users to calculate the rate of spread and direction of
a fire across a triangle, given three time measurements and details about
the orientation and distance between observational points. The algorithm is
based on the description from Simard et al. (1984). See \code{pros} for more
information.

The functions require the user to arrange the input dataframe so that each
triangle of interest is identified based on a new row in the dataframe. The
input format forces the user to identify the triangles, one triangle per row
of input dataframe. Very complex arrangements of field plot layouts are
possible, and the current version of these functions do not attempt to
determine each triangle of interest automatically.
\end{Details}
%
\begin{Value}
\code{lros} returns a dataframe which includes the rate of spread
and spread direction. Output units depend on the user’s inputs for
distance (typically meters) and time (seconds or minutes).
\end{Value}
%
\begin{Author}\relax
Tom Schiks, Xianli Wang, Alan Cantin
\end{Author}
%
\begin{References}\relax
1. Simard, A.J., Eenigenburg, J.E., Adams, K.B., Nissen, R.L.,
Deacon, and Deacon, A.G. 1984. A general procedure for sampling and
analyzing wildland fire spread.

2. Byram, G.M. 1959. Combustion of forest fuels. In: Davis, K.P. Forest Fire
Control and Use. McGraw-Hill, New York.

3. Curry, J.R., and Fons, W.L. 1938. Rate of spread of surface fires in the
Ponderosa Pine Type of California. Journal of Agricultural Research 57(4):
239-267.

4. Simard, A.J., Deacon, A.G., and Adams, K.B. 1982. Nondirectional sampling
wildland fire spread. Fire Technology: 221-228.
\end{References}
%
\begin{SeeAlso}\relax
\code{\LinkA{pros}{pros}},
\end{SeeAlso}
%
\begin{Examples}
\begin{ExampleCode}

library(cffdrs)
# manual single entry, but converted to a data frame
lros.in1 <- data.frame(t(c(0, 24.5, 50, 22.6, 120, 20.0, 90, 35)))
colnames(lros.in1)<-c("T1","LengthT1T2", "T2", "LengthT1T3", "T3", 
                      "LengthT2T3", "bearingT1T2", "bearingT1T3")
lros.out1 <- lros(lros.in1)
lros.out1

# multiple entries using a dataframe
# load the test dataframe for lros
data("test_lros")
lros(test_lros)



\end{ExampleCode}
\end{Examples}
\inputencoding{utf8}
\HeaderA{pros}{Point-based input for Simard Rate of Spread and Direction}{pros}
\keyword{ros}{pros}
%
\begin{Description}\relax
\code{pros} is used to calculate the rate of spread and direction given one
set of three point-based observations of fire arrival time. The function
requires that the user specify the time that the fire crossed each point,
along with the latitude and longitude of each observational point. This
function allows quick input of a dataframe specifying one or many triangles.
\end{Description}
%
\begin{Usage}
\begin{verbatim}
pros(input)
\end{verbatim}
\end{Usage}
%
\begin{Arguments}
\begin{ldescription}
\item[\code{input}] A dataframe containing input variables of Time fire front
crossed points 1, 2, 3, and latitude/longitude for those same points.
Variable names have to be the same as in the following list, but they are
case insensitive. The order in which the input variables are entered is not
important.


\Tabular{lll}{ 
\var{T1} & (required) & Time that the fire front
crossed point 1. Time entered in fractional \\{}&& format. Output ROS
will depend on the level of precision entered \\{}&& (minute, second,
decisecond)\\{} 
\var{T2} & (required) & Time that the fire front
crossed point 2. Time entered in fractional \\{}&& format. Output ROS
will depend on the level of precision entered \\{}&& (minute, second,
decisecond)\\{} 
\var{T3} & (required) & Time that the fire front crossed point 3. Time 
entered in fractional \\{}&& format. Output ROS will depend on the level
of precision entered \\{}&& (minute, second,
decisecond)\\{} 
\var{Long1}& (required) & Longitude for datalogger 1. (decimal degrees). \\{} 
\var{Long2}& (required) & Longitude for datalogger 2. (decimal degrees). \\{} 
\var{Long3}& (required) & Longitude for datalogger 3. (decimal degrees). \\{} 
\var{Lat1} & (required) & Latitude for datalogger 1. (decimal degrees). \\{} 
\var{Lat2} & (required) & Latitude for datalogger 2. (decimal degrees). \\{}
\var{Lat3} & (required) & Latitude for datalogger 3. (decimal
degrees). \\{} }
\end{ldescription}
\end{Arguments}
%
\begin{Details}\relax
\code{pros} allows R users to calculate the rate of spread and direction of
a fire across a triangle, given three time measurements and details about
the orientation and distance between observational points. The algorithm is
based on the description from Simard et al. (1984).

Rate of spread and direction of spread are primary variables of interest
when observing wildfire growth over time. Observations might be recorded
during normal fire management operations (e.g., by a Fire Behaviour
Analyst), during prescribed fire treatments, and during experimental
research burns. Rate of spread is especially important for estimating
Byram's fireline intensity, fireline intensity = heat constant of fuel ×
weight of fuel consumed × forward rate of spread (Byram 1959).

Rate of spread is difficult to measure and highly variable in the field.
Many techniques were proposed over the years, but most were based on
observations collected from a pre-placed reference grid and stopwatch (Curry
and Fons 1938; Simard et al. 1982). Early approaches required that observers
be in visual contact with the reference grid, but later, thermocouples and
dataloggers were employed to measure the onset of the heat pulse at each
point.

Simard et al. (1982) proposed calculations for spread based on an
equilateral triangle layout. Simard et al. (1984) proposed calculations for
spread based on any type of triangle. Both articles also discussed field
sampling design and layout, with special attention to the size of the
triangles (large enough that the fire traverses the triangle in one to two
minutes) and even using triangles of varying size within one field plot (but
no triangle larger than one fourth of the site's total area).

The underlying algorithms use trigonometry to solve for rate of spread and
direction of spread. One important assumption is that the spread rate and
direction is uniform across one triangular plot, and that the fire front is
spreading as a straight line; Simard et al. (1982, 1984) acknowledge that
these assumption are likely broken to some degree during fire spread events.

The functions require the user to arrange the input dataframe so that each
triangle of interest is identified based on a new row in the dataframe. The
input format forces the user to identify the triangles, one triangle per row
of input dataframe. Very complex arrangements of field plot layouts are
possible, and the current version of these functions do not attempt to
determine each triangle of interest automatically.
\end{Details}
%
\begin{Value}
\code{pros} returns a dataframe which includes the rate of spread
and spread direction. Output units depend on the user’s inputs for
distance (typically meters) and time (seconds or minutes).
\end{Value}
%
\begin{Author}\relax
Tom Schiks, Xianli Wang, Alan Cantin
\end{Author}
%
\begin{References}\relax
1. Simard, A.J., Eenigenburg, J.E., Adams, K.B., Nissen, R.L.,
Deacon, and Deacon, A.G. 1984. A general procedure for sampling and
analyzing wildland fire spread.

2. Byram, G.M. 1959. Combustion of forest fuels. In: Davis, K.P. Forest Fire
Control and Use. McGraw-Hill, New York.

3. Curry, J.R., and Fons, W.L. 1938. Rate of spread of surface fires in the
Ponderosa Pine Type of California. Journal of Agricultural Research 57(4):
239-267.

4. Simard, A.J., Deacon, A.G., and Adams, K.B. 1982. Nondirectional sampling
wildland fire spread. Fire Technology: 221-228.
\end{References}
%
\begin{SeeAlso}\relax
\code{\LinkA{lros}{lros}},
\end{SeeAlso}
%
\begin{Examples}
\begin{ExampleCode}

library(cffdrs)
# manual single entry
pros.in1 <- data.frame(t(c(2, -79.701027, 43.808872, 50, -79.699650, 43.808833
                            , 120, -79.700387, 43.809816)))
colnames(pros.in1)<-c("T1", "LONG1", "LAT1", "T2", "LONG2", "LAT2", "T3", 
                      "LONG3", "LAT3")
pros.out1 <- pros(pros.in1)
# multiple entries using a dataframe
# load the test dataframe for pros
data("test_pros")
pros(test_pros)

\end{ExampleCode}
\end{Examples}
\inputencoding{utf8}
\HeaderA{sdmc}{Sheltered Duff Moisture Code}{sdmc}
\keyword{methods}{sdmc}
%
\begin{Description}\relax
\code{sdmc} is used to calculate sheltered DMC (sDMC, Wotton et al., 2005)
based on daily noon weather observations of temperature, relative humidity,
wind speed, 24-hour rainfall, and a previous day's calculated or estimated
value of sDMC. This function calculates sDMC for either one weather station
or for multiple weather stations over the duration of the daily weather data
set, typically over a fire season.
\end{Description}
%
\begin{Usage}
\begin{verbatim}
sdmc(input, sdmc_old = NULL, batch = TRUE)
\end{verbatim}
\end{Usage}
%
\begin{Arguments}
\begin{ldescription}
\item[\code{input}] A data.frame containing input variables of daily noon weather
observations. Variable names have to be the same as in the following list,
but they are case insensitive. The order in which the input variables are
entered is not important either.


\Tabular{lll}{ 
\var{temp} & (required) & Temperature (centigrade)\\{}
\var{rh} & (required) & Relative humidity (\%)\\{} 
\var{ws} & (required) & 10-m height wind speed (km/h)\\{} 
\var{prec} & (required) & 1-hour rainfall (mm)\\{}
\var{mon} & (recommended) & Month of the observations (integer 1-12)\\{} 
\var{day} & (optional) & Day of the observations (integer)\\{} }

\item[\code{sdmc\_old}] Previous day's value of SDMC. At the start of calculations,
when there is no calculated previous day's SDMC value to use, the user must
specify an estimate of this value.  Where \code{sdmc\_old=NULL}, the function
will calculate the initial SDMC values based on the initial DMC. The
\code{sdmc\_old} argument can accept a single initial value for multiple
weather stations, and also accept a vector of initial values for multiple
weather stations.

\item[\code{batch}] Whether the computation is iterative or single step, default is
TRUE. When \code{batch=TRUE}, the function will calculate daily SDMC for one
weather station over a period of time iteratively. If multiple weather
stations are processed, an additional "id" column is required in the input
to label different stations, and the data needs to be sorted by date/time
and "id".  If \code{batch=FALSE}, the function calculates only one time step
base on either the previous day's SDMC or the initial start value.
\end{ldescription}
\end{Arguments}
%
\begin{Details}\relax
The Duff Moisture Code (DMC) component of the Canadian Forest Fire Weather
Index (FWI) System tracks moisture content of the forest floor away from the
sheltering influences of overstory trees.  This sheltered Duff Moisture Code
(sDMC) was developed to track moisture in the upper 5 cm of the organic
layer in the rain sheltered areas near (<0.5 m) the boles of overstory trees
(Wotton et al. 2005), an area where lightning strikes usually ignite the
forest floor when they run to ground. The sDMC is very similar in structure
(and identical in data requirements) to the DMC.  The sDMC, like all the FWI
System moisture codes, is a bookkeeping system that tracks gain and loss of
moisture from day-to-day; thus an estimate of the previous day's sDMC value
is needed to provide a starting point for each day's moisture calculation.
Like the other moisture codes in the FWI System the sDMC is converted from a
moisture content value to an outputted CODE value which increases in value
with decreasing moisture content.
\end{Details}
%
\begin{Value}
\code{sdmc} returns either a single value or a vector of SDMC
values.
\end{Value}
%
\begin{Author}\relax
Xianli Wang, Mike Wotton, Alan Cantin, and Mike Flannigan
\end{Author}
%
\begin{References}\relax
Wotton, B.M., B.J. Stocks, and D.L. Martell. 2005. An index for
tracking sheltered forest floor moisture within the Canadian Forest Fire
Weather Index System. International Journal of Wildland Fire, 14, 169-182.
\end{References}
%
\begin{SeeAlso}\relax
\code{\LinkA{fwi}{fwi}}
\end{SeeAlso}
%
\begin{Examples}
\begin{ExampleCode}

library(cffdrs)
data("test_sdmc")
#order the data:
test_sdmc<-test_sdmc[with(test_sdmc,order(yr,mon,day)),]
# (1)Default of sdmc, calculate sdmc for a chronical period 
# of time. 
# Because sdmc_old is better to be calculated, we normally
# ignore this option:
test_sdmc$SDMC<-sdmc(test_sdmc)
# (2) multiple weather stations: 
# Batch process with multiple stations (2 stations) assuming
# they are from the same month:
test_sdmc$mon<-7
test_sdmc$day<-rep(1:24,2)
test_sdmc$id<-rep(1:2,each=24)
# Sort the data by date and weather station id:
test_sdmc<-test_sdmc[with(test_sdmc,order(yr,mon,day,id)),]
# Apply the function
test_sdmc$SDMC_mult_stn<-sdmc(test_sdmc,batch=TRUE)
# Assuming each record is from a different weather station, and 
# calculate only one time step: 
  foo<-sdmc(test_sdmc,batch=FALSE)

\end{ExampleCode}
\end{Examples}
\inputencoding{utf8}
\HeaderA{test\_fbp}{Fire Behaviour Prediction Sample Data Set}{test.Rul.fbp}
\keyword{datasets}{test\_fbp}
%
\begin{Description}\relax
This data set is a set of input data for each of the test cases in the
publication supplied below.
\end{Description}
%
\begin{Format}
A data frame containing 24 columns, 21 rows, including 1 header line
\end{Format}
%
\begin{Source}\relax
\url{http://cfs.nrcan.gc.ca/pubwarehouse/pdfs/31414.pdf}
\end{Source}
%
\begin{References}\relax
1. Wotton, B.M., Alexander, M.E., Taylor, S.W. 2009. Updates and
revisions to the 1992 Canadian forest fire behavior prediction system. Nat.
Resour. Can., Can. For. Serv., Great Lakes For. Cent., Sault Ste. Marie,
Ontario, Canada. Information Report GLC-X-10, 45p.
\end{References}
\inputencoding{utf8}
\HeaderA{test\_fbpRaster}{Raster Data for fbpRaster function}{test.Rul.fbpRaster}
\keyword{datasets}{test\_fbpRaster}
%
\begin{Description}\relax
Test raster file to calculate fbp data.
\end{Description}
%
\begin{Format}
A raster (tif) file.
\end{Format}
\inputencoding{utf8}
\HeaderA{test\_fwi}{Fire Weather Index Sample Input Data Set}{test.Rul.fwi}
\keyword{datasets}{test\_fwi}
%
\begin{Description}\relax
This data set is the sample input data that was used in original FWI program
calibration.
\end{Description}
%
\begin{Format}
A data frame containing 9 columns and 49 rows, with 1 header line
\end{Format}
%
\begin{Source}\relax
\url{http://cfs.nrcan.gc.ca/pubwarehouse/pdfs/19973.pdf}
\end{Source}
%
\begin{References}\relax
1. Van Wagner, CE. and T.L. Pickett. 1985. Equations and FORTRAN
program for the Canadian Forest Fire Weather Index System. Can. For. Serv.,
Ottawa, Ont. For. Tech. Rep. 33. 18 p.
\end{References}
\inputencoding{utf8}
\HeaderA{test\_gfmc}{Grass Fuel Moisture Code Sample Input Data Set}{test.Rul.gfmc}
\keyword{datasets}{test\_gfmc}
%
\begin{Description}\relax
This data set is the sample input data that was used in original FWI program
calibration.
\end{Description}
%
\begin{Format}
A data frame containing 9 columns and 199 rows, with 1 header line
\end{Format}
\inputencoding{utf8}
\HeaderA{test\_hffmc}{Hourly Fine Fuel Moisture Code Sample Input Data Set}{test.Rul.hffmc}
\keyword{datasets}{test\_hffmc}
%
\begin{Description}\relax
Sample dataset for use with the \code{hffmc} function.
\end{Description}
%
\begin{Format}
A data frame containing 8 columns and 481 rows, including 1 header
line
\end{Format}
\inputencoding{utf8}
\HeaderA{test\_lros}{Line-based Simard function Sample Data Set}{test.Rul.lros}
\keyword{datasets}{test\_lros}
\keyword{lros}{test\_lros}
\keyword{simard}{test\_lros}
%
\begin{Description}\relax
This is a set of input data to test the lros function.
\end{Description}
%
\begin{Format}
A data frame containing 8 columns, 4 rows, including 1 header line.
\end{Format}
%
\begin{Source}\relax
no source
\end{Source}
%
\begin{References}\relax
1. Simard, A.J., Eenigenburg, J.E., Adams, K.B., Nissen, R.L.,
Deacon, and Deacon, A.G. 1984. A general procedure for sampling and
analyzing wildland fire spread.

2. Byram, G.M. 1959. Combustion of forest fuels. In: Davis, K.P. Forest Fire
Control and Use. McGraw-Hill, New York.

3. Curry, J.R., and Fons, W.L. 1938. Rate of spread of surface fires in the
Ponderosa Pine Type of California. Journal of Agricultural Research 57(4):
239-267.

4. Simard, A.J., Deacon, A.G., and Adams, K.B. 1982. Nondirectional sampling
wildland fire spread. Fire Technology: 221-228.
\end{References}
\inputencoding{utf8}
\HeaderA{test\_pros}{Point-based Simard function Sample Data Set}{test.Rul.pros}
\keyword{datasets}{test\_pros}
\keyword{lros}{test\_pros}
\keyword{simard}{test\_pros}
%
\begin{Description}\relax
This is a set of input data to test the pros function.
\end{Description}
%
\begin{Format}
A data frame containing 9 columns, 4 rows, including 1 header line.
\end{Format}
%
\begin{Source}\relax
no source
\end{Source}
%
\begin{References}\relax
1. Simard, A.J., Eenigenburg, J.E., Adams, K.B., Nissen, R.L.,
Deacon, and Deacon, A.G. 1984. A general procedure for sampling and
analyzing wildland fire spread.

2. Byram, G.M. 1959. Combustion of forest fuels. In: Davis, K.P. Forest Fire
Control and Use. McGraw-Hill, New York.

3. Curry, J.R., and Fons, W.L. 1938. Rate of spread of surface fires in the
Ponderosa Pine Type of California. Journal of Agricultural Research 57(4):
239-267.

4. Simard, A.J., Deacon, A.G., and Adams, K.B. 1982. Nondirectional sampling
wildland fire spread. Fire Technology: 221-228.
\end{References}
\inputencoding{utf8}
\HeaderA{test\_rast\_day01}{Raster Data for fwiRaster function}{test.Rul.rast.Rul.day01}
\keyword{datasets}{test\_rast\_day01}
%
\begin{Description}\relax
Daily fire weather inputs obtained from the Global Environmental Multiscale
Model (GEM) in northern Alberta
\end{Description}
%
\begin{Format}
A raster (tif) file.
\end{Format}
\inputencoding{utf8}
\HeaderA{test\_rast\_day02}{Raster Data for fwiRaster function}{test.Rul.rast.Rul.day02}
\keyword{datasets}{test\_rast\_day02}
%
\begin{Description}\relax
Daily fire weather inputs obtained from the Global Environmental Multiscale
Model (GEM) in northern Alberta
\end{Description}
%
\begin{Format}
A raster (tif) file.
\end{Format}
\inputencoding{utf8}
\HeaderA{test\_rast\_hour01}{Raster Data for ffmcRaster function}{test.Rul.rast.Rul.hour01}
\keyword{datasets}{test\_rast\_hour01}
%
\begin{Description}\relax
Hourly fire weather inputs obtained from the Global Environmental Multiscale
Model (GEM) in northern Alberta
\end{Description}
%
\begin{Format}
A raster (tif) file.
\end{Format}
\inputencoding{utf8}
\HeaderA{test\_rast\_hour02}{Raster Data for ffmcRaster function}{test.Rul.rast.Rul.hour02}
\keyword{datasets}{test\_rast\_hour02}
%
\begin{Description}\relax
Hourly fire weather inputs obtained from the Global Environmental Multiscale
Model (GEM) in northern Alberta
\end{Description}
%
\begin{Format}
A raster (tif) file.
\end{Format}
\inputencoding{utf8}
\HeaderA{test\_sdmc}{Sheltered Duff Moisture Code Sample Input Data Set}{test.Rul.sdmc}
\keyword{datasets}{test\_sdmc}
%
\begin{Description}\relax
This data set is the sample input data that was used in original FWI program
calibration, but with an initial dmc value populated.
\end{Description}
%
\begin{Format}
A data frame containing 10 columns and 49 rows, including 1 header
line
\end{Format}
%
\begin{Source}\relax
\url{http://cfs.nrcan.gc.ca/pubwarehouse/pdfs/19973.pdf}
\end{Source}
%
\begin{References}\relax
1. Van Wagner, CE. and T.L. Pickett. 1985. Equations and FORTRAN
program for the Canadian Forest Fire Weather Index System. Can. For. Serv.,
Ottawa, Ont. For. Tech. Rep. 33. 18 p.
\end{References}
\inputencoding{utf8}
\HeaderA{test\_wDC}{Overwinter Drought Code Sample Input Data Set}{test.Rul.wDC}
\keyword{datasets}{test\_wDC}
%
\begin{Description}\relax
This dataset has 2 ID values (weather stations), and each have 2 sequential
years. This data can be used as an example to calculated overwintered DC.
There are 10 columns and 1463 rows, including 1 header row.
\end{Description}
%
\begin{Format}
A data frame containing 10 columns and 1463 rows, including 1 header
line
\end{Format}
\inputencoding{utf8}
\HeaderA{test\_wDC\_fs}{Fire Season Dataset to test Overwinter Drought Code}{test.Rul.wDC.Rul.fs}
\keyword{datasets}{test\_wDC\_fs}
%
\begin{Description}\relax
This dataset has pre-set start and end dates to the fire season for 2
weather stations. The point of this dataset is to demonstrate that a data
frame of start and end dates for the fire season can be calculated and
applied to the program.
\end{Description}
%
\begin{Format}
A data frame containing 7 columns and 9 rows, including 1 header
line
\end{Format}
\inputencoding{utf8}
\HeaderA{wDC}{Overwintering Drought Code}{wDC}
\keyword{methods}{wDC}
%
\begin{Description}\relax
\code{wDC} calculates an initial or season starting Drought Code (DC) value
based on a standard method of overwintering the Drought Code (Lawson and
Armitage 2008).  This method uses the final DC value from previous year,
over winter precipitation and estimates of how much over-winter
precipitation 'refills' the moisture in this fuel layer. This function could
be used for either one weather station or for multiple weather stations.
\end{Description}
%
\begin{Usage}
\begin{verbatim}
wDC(DCf = 100, rw = 200, a = 0.75, b = 0.75)
\end{verbatim}
\end{Usage}
%
\begin{Arguments}
\begin{ldescription}
\item[\code{DCf}] Final fall DC value from previous year

\item[\code{rw}] Winter precipitation (mm)

\item[\code{a}] User selected values accounting for carry-over fraction (view table
below)

\item[\code{b}] User selected values accountain for wetting efficiency fraction
(view table below)
\end{ldescription}
\end{Arguments}
%
\begin{Details}\relax
Of the three fuel moisture codes (i.e. FFMC, DMC and DC) making up the FWI
System, only the DC needs to be considered in terms of its values carrying
over from one fire season to the next.  In Canada both the FFMC and the DMC
are assumed to reach moisture saturation from overwinter precipitation at or
before spring melt; this is a reasonable assumption and any error in these
assumed starting conditions quickly disappears.  If snowfall (or other
overwinter precipitation) is not large enough however, the fuel layer
tracked by the Drought Code may not fully reach saturation after spring snow
melt; because of the long response time in this fuel layer (53 days in
standard conditions) a large error in this spring starting condition can
affect the DC for a significant portion of the fire season.  In areas where
overwinter precipitation is 200 mm or more, full moisture recharge occurs
and DC overwintering is usually unnecessary.  More discussion of
overwintering and fuel drying time lag can be found in Lawson and Armitage
(2008) and Van Wagner (1985).
\end{Details}
%
\begin{Value}
\code{wDC} returns either a single value or a vector of wDC values.
\end{Value}
%
\begin{Author}\relax
Xianli Wang, Mike Wotton, Alan Cantin, and Mike Flannigan
\end{Author}
%
\begin{References}\relax
Lawson B.D. and Armitage O.B. 2008. Weather Guide for the
Canadian Forest Fire Danger Rating System. Natural Resources Canada,
Canadian Forest Service, Northern Forestry Centre, Edmonton, Alberta. 84 p.
\url{http://cfs.nrcan.gc.ca/pubwarehouse/pdfs/29152.pdf}

Van Wagner, C.E. 1985. Drought, timelag and fire danger rating. Pages
178-185 in L.R. Donoghue and R.E. Martin, eds. Proc. 8th Conf. Fire For.
Meteorol., 29 Apr.-3 May 1985, Detroit, MI. Soc. Am. For., Bethesda, MD.
\url{http://cfs.nrcan.gc.ca/pubwarehouse/pdfs/23550.pdf}
\end{References}
%
\begin{SeeAlso}\relax
\code{\LinkA{fwi}{fwi}}, \code{\LinkA{fireSeason}{fireSeason}}
\end{SeeAlso}
%
\begin{Examples}
\begin{ExampleCode}

library(cffdrs)
# The standard test data:
data("test_wDC")
# (1) Simple case previous fall's DC was 300, overwinter 
# rain 110mm
winter_DC <- wDC(DCf=300,rw=110)
winter_DC
#(2) modified a and b parameters. Find table values in listed 
# reference for Lawson and Armitage, 2008.
winter_DC <- wDC(DCf=300,rw=110,a=1.0,b=0.9)
winter_DC
#(3)with multiple inputs:
winter_DC <- wDC(DCf=c(400,300,250), rw=c(99,110,200),
                   a=c(0.75,1.0,0.75), b=c(0.75,0.9,0.75))
winter_DC
#(4) A realistic example:
#precipitation accumulation and date boundaries
input <- test_wDC
#order data by ID and date
input <- with(input,input[order(id,yr,mon,day),])
input$date <- as.Date(as.POSIXlt(paste(input$yr,"-",input$mon,"-",input$day,sep="")))
#select id value 1
input.2 <- input[input$id==2,]
#Explicitly defined fire start and end dates.
data("test_wDC_fs")
print(test_wDC_fs)
#Set date field
test_wDC_fs$date <- as.Date(as.POSIXlt(paste(test_wDC_fs$yr,"-",test_wDC_fs$mon,"-",
                                             test_wDC_fs$day,sep="")))
#match to current id value
input.2.fs <- test_wDC_fs[test_wDC_fs$id==2,]
#assign start of winter date (or end of fire season date)
winterStartDate <- input.2.fs[2,"date"]
#assign end of winter date (or start of new fire season date)
winterEndDate <-  input.2.fs[3,"date"]
#Accumulate overwinter precip based on chosen dates
curYr.prec <- sum(input.2[(input.2$date>winterStartDate & input.2$date < winterEndDate),]$prec)
#Assign a fall DC value
fallDC <- 500
#calculate winter DC
winter_DC <- wDC(DCf=fallDC,rw=curYr.prec)
winter_DC
#Assign a different fall DC value
fallDC <- 250
#calculate winter DC
winter_DC <- wDC(DCf=fallDC,rw=curYr.prec,a=1.0)
winter_DC

\end{ExampleCode}
\end{Examples}
\printindex{}
\end{document}
